\documentclass{report} 
\title{Strogatz}
\date{Started 20th November 2025}
\author{Malcolm}
\usepackage{amsmath} %import math
\usepackage{mathtools} %more math
\usepackage{amssymb} %for QED symbol
\usepackage{amsthm} %
\usepackage{mathrsfs} %
\usepackage{bm}%bold math
\usepackage{graphicx} %import imaging

\graphicspath{{./images/}} %set imaging path
\newtheorem*{theorem}{Theorem}
\theoremstyle{definition}
\newtheorem*{definition}{Definition}


\begin{document}
\maketitle

\tableofcontents

\chapter{Flows on a line}
\section{Introduction}
We have the general system as
\begin{align*}
\dot x_1=&f_1(x_1,\ldots,x_n)\\
&\vdots\\
\dot x_n=&f_n(x_1,\ldots,x_n)
\end{align*}
its solutions can be intuitively visualised as trajectories flowing through an $n$-dimensional phase space. In these early sections we start with the simple case where $n=1$; then we get a single
equation of the form
\begin{equation*}
\dot x=f(x)
\end{equation*}
We call such systems one-dimensional or first-order systems.\\
\vspace{1mm}\\
We clarify the following:
\begin{enumerate}
\item The word \textit{system} is being used here in the context of a dynamical system, not in the classical sense of a collection of two or more equations. Thus a single equation can be called a
`system'.
\item We do not allow $f$ to depend explicitly on time. Time-dependent or `nonautonomous' equations of the form $\dot x=f(x,t)$ are more complicated because they require two variables. Such a system
in this case would be regarded as a two-dimensional system, to be discussed later.
\end{enumerate}
\newpage

\section{Geometric intuition}
\subsection{Instructive example}
Consider the following nonlinear differential equation:
\begin{equation*}
\dot x=\sin x
\end{equation*}
This particular equation can be solved in closed form; using separation of variables then integration we get
\begin{equation*}
t=\ln\left|\frac{\csc x_0+\cot x_0}{\csc x+\cot x}\right|
\end{equation*}
where $x=x_0$ at $t=0$. Although the result is exact, it isn't really possible to interpret. For instance consider the following
\begin{enumerate}
\item Suppose $x_0=\pi/4$. Describe the qualitative features of the solution $x(t)$ for all $t>0$. What happens as $t\to\infty$?
\item For an arbitrary initial condition $x_0$, what is the behaviour of $x(t)$ as $t\to\infty$?
\end{enumerate}
The formula above is not transparent. We consider instead a graphical analysis; the differential equation $\dot x=\sin x$ represents a \textit{vector field} on the line: it dictates the velocity
vector $\dot x$ at each $x$. To sketch this vector field, it is convenient to plot $\dot x$ vs $x$, drawing arrows on the $x$-axis to indicate the corresponding velocity vector at each $x$.
The arrows point to the right when $\dot x>0$ and to the left when $\dot x<0$:
\begin{center}
\includegraphics[width=10cm]{1}
\end{center}
The \textit{flow} is to the right when $\dot x>0$ and to the left when $\dot x<0$; at points where $\dot x=0$, there is no flow, and such points are therefore called \textit{fixed points}. Solid
dots represent \textit{stable} fixed points (often called \textit{attractors/sinks}, bscause the flow is toward them). Open dots represent \textit{unstable} fixed points (also known as \textit{repellers/sources}).\\
(next page)\newpage
\noindent\textbf{Cont.}\\
We had
\begin{center}
\includegraphics[width=9cm]{1}
\end{center}
We start at $x_0$ and watch how $x$ is carried along by the flow. Consider now the questions proposed earlier.
\begin{enumerate}
\item Starting at $x_0=\pi/4$, the flow moves to the right faster and faster until $x$ crosses $\pi/2$ (where $\sin x$ reaches its maximum). Then the flow slows down and eventually approaches
the stable fixed point $x=\pi$ from the left. As such the qualitative form of the solution is as follows
\begin{center}
\includegraphics[width=7cm]{2}
\end{center}
See that the curve is concave up at first, then concave down; this change in concavity corresponds to the initial acceleration for $x<\pi/2$, followed by the deceleration toward $x=\pi$.\\
(next page)\newpage
\item The same reasoning applies to any initial condition $x_0$. If $\dot x>0$ initially, the flow is to the right and $x$ asymptotically approaches the nearest stable point, and similarly to the left for the
other case where $\dot x<0$. If $\dot x=0$, then $x$ remains constant. The qualitative form of the solution for any initial condition can be sketched: 
\begin{center}
\includegraphics[width=6cm]{3}
\end{center}
\end{enumerate}
\noindent Note that the picture can't tell us certain \textit{quantitative} things: for instance we don't know the time at which the speed $|\dot x|$ is greatest. But in many cases the \textit{qualitative}
information is what we care about. In this case these sketches are useful.
\newpage

\subsection{Fixed points and stability}















\end{document}
