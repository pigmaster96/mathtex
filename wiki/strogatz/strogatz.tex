\documentclass{report} 
\title{Strogatz}
\date{Started 20th November 2025}
\author{Malcolm}
\usepackage{amsmath} %import math
\usepackage{mathtools} %more math
\usepackage{amssymb} %for QED symbol
\usepackage{amsthm} %
\usepackage{mathrsfs} %
\usepackage{bm}%bold math
\usepackage{graphicx} %import imaging

\graphicspath{{./images/}} %set imaging path
\newtheorem*{theorem}{Theorem}
\theoremstyle{definition}
\newtheorem*{definition}{Definition}


\begin{document}
\maketitle

\tableofcontents

\chapter{Flows on a line}
\section{Introduction}
We have the general system as
\begin{align*}
\dot x_1=&f_1(x_1,\ldots,x_n)\\
&\vdots\\
\dot x_n=&f_n(x_1,\ldots,x_n)
\end{align*}
its solutions can be intuitively visualised as trajectories flowing through an $n$-dimensional phase space. In these early sections we start with the simple case where $n=1$; then we get a single
equation of the form
\begin{equation*}
\dot x=f(x)
\end{equation*}
We call such systems one-dimensional or first-order systems.\\
\vspace{1mm}\\
We clarify the following:
\begin{enumerate}
\item The word \textit{system} is being used here in the context of a dynamical system, not in the classical sense of a collection of two or more equations. Thus a single equation can be called a
`system'.
\item We do not allow $f$ to depend explicitly on time. Time-dependent or `nonautonomous' equations of the form $\dot x=f(x,t)$ are more complicated because they require two variables. Such a system
in this case would be regarded as a two-dimensional system, to be discussed later.
\end{enumerate}
\newpage

\section{Geometric intuition}
\subsection{Instructive example}
Consider the following nonlinear differential equation:
\begin{equation*}
\dot x=\sin x
\end{equation*}
This particular equation can be solved in closed form; using separation of variables then integration we get
\begin{equation*}
t=\ln\left|\frac{\csc x_0+\cot x_0}{\csc x+\cot x}\right|
\end{equation*}
where $x=x_0$ at $t=0$. Although the result is exact, it isn't really possible to interpret. For instance consider the following
\begin{enumerate}
\item Suppose $x_0=\pi/4$. Describe the qualitative features of the solution $x(t)$ for all $t>0$. What happens as $t\to\infty$?
\item For an arbitrary initial condition $x_0$, what is the behaviour of $x(t)$ as $t\to\infty$?
\end{enumerate}
The formula above is not transparent. We consider instead a graphical analysis; the differential equation $\dot x=\sin x$ represents a \textit{vector field} on the line: it dictates the velocity
vector $\dot x$ at each $x$. To sketch this vector field, it is convenient to plot $\dot x$ vs $x$, drawing arrows on the $x$-axis to indicate the corresponding velocity vector at each $x$.
The arrows point to the right when $\dot x>0$ and to the left when $\dot x<0$:
\begin{center}
\includegraphics[width=10cm]{1}
\end{center}
The \textit{flow} is to the right when $\dot x>0$ and to the left when $\dot x<0$; at points where $\dot x=0$, there is no flow, and such points are therefore called \textit{fixed points}. Solid
dots represent \textit{stable} fixed points (often called \textit{attractors/sinks}, bscause the flow is toward them). Open dots represent \textit{unstable} fixed points (also known as \textit{repellers/sources}).\\
(next page)\newpage
\noindent\textbf{Cont.}\\
We had
\begin{center}
\includegraphics[width=9cm]{1}
\end{center}
We start at $x_0$ and watch how $x$ is carried along by the flow. Consider now the questions proposed earlier.
\begin{enumerate}
\item Starting at $x_0=\pi/4$, the flow moves to the right faster and faster until $x$ crosses $\pi/2$ (where $\sin x$ reaches its maximum). Then the flow slows down and eventually approaches
the stable fixed point $x=\pi$ from the left. As such the qualitative form of the solution is as follows
\begin{center}
\includegraphics[width=7cm]{2}
\end{center}
See that the curve is concave up at first, then concave down; this change in concavity corresponds to the initial acceleration for $x<\pi/2$, followed by the deceleration toward $x=\pi$.\\
(next page)\newpage
\item The same reasoning applies to any initial condition $x_0$. If $\dot x>0$ initially, the flow is to the right and $x$ asymptotically approaches the nearest stable point, and similarly to the left for the
other case where $\dot x<0$. If $\dot x=0$, then $x$ remains constant. The qualitative form of the solution for any initial condition can be sketched: 
\begin{center}
\includegraphics[width=6cm]{3}
\end{center}
\end{enumerate}
\noindent Note that the picture can't tell us certain \textit{quantitative} things: for instance we don't know the time at which the speed $|\dot x|$ is greatest. But in many cases the \textit{qualitative}
information is what we care about. In this case these sketches are useful.
\newpage

\subsection{Fixed points and stability}
The ideas discussed can be extended to any one-dimensional system $\dot x=f(x)$. We draw the graph of $f(x)$.
\begin{center}
\includegraphics[width=8cm]{4}
\end{center}
As before, the flow is to the right where $f(x)>0$ and to the left when $f(x)<0$. To find the solution to $\dot x=f(x)$ starting from an arbitrary initial condition $x_0$, we start at $x=x_0$ and 
watch how it is carried along by the flow.\\
\vspace{1mm}\\
The point moves along the $x$ axis according to some function $x(t)$. This function is called the \textit{trajectory} based at $x_0$, and it represents the solution of the differential equation
starting from the initial condition $x_0$. A figure like the one above, which shows all the different trajectories of the system, is called a \textit{phase portrait}.\\
\vspace{1mm}\\
The appearance of the phase portrait is controlled by the fixed points $x^*$, defined by $f(x^*)=0$; they correspond to stagnation points of flow. As shown earlier the solid dots represent stable
fixed points while open dots represent unstable fixed points. In terms of trajectories, fixed points represent \textit{equilibrium} solutions (sometimes called steady/constant/rest solutions), 
since if $x=x^*$ initially it stays there for all of time.\\
\vspace{1mm}\\
An equilibrium is defined to be stable if all sufficiently small disturbances away from it damp out in time, so stable equilibria are 
represented geometrically by stable fixed points. Conversely, unstable equilibria, in which disturbances grow over time, are represented by unstable points.\\
(next page)\newpage
\noindent\textbf{Instructive examples}\\
Say we want to find all the fixed points for $\dot x=x^2-1$, and then classify their stability.\\
\vspace{1mm}\\
Here $f(x)=x^2-1$. To find the fixed points we set $f(x^*)=0$ and solve for $x^*$; thus $x^*=\pm1$. To determine stability we plot $x^2-1$ and then sketch the vector field:
\begin{center}
\includegraphics[width=7cm]{5}
\end{center}
The flow is to the right when $x^2-1>0$ and to the left where $x^2-1<0$. $x^*=-1$ is stable and $x^*=1$ is unstable.\\
\vspace{1mm}\\
Note that the definition of stable equilibrium is based on \textit{small} disturbances; certain large disturbances may fail to decay. In this case see that all small disturbances to 
$x^*=-1$ will decay, but a large disturbance that sends $x$ to the right of $x=1$ will \textit{not} decay---and will in fact be repelled out to $+\infty$. To emphasise this we sometimes say that
$x^*=-1$ is \textit{locally stable}, but not globally stable.\\
\vspace{1mm}\\
On the other hand consider the vector field for the RC circuit equation
\begin{equation*}
\dot Q=f(Q)=\frac{V_0}{R}-\frac{Q}{RC}
\end{equation*}
\begin{center}
\includegraphics[width=5cm]{6}
\end{center}
In this case the flow is always towards $Q^*$; it is a \textit{stable} fixed point. In fact, it is \textit{globally stable}, in the sense that it is approached from \textit{all} initial 
conditions.\\
(next page)\newpage
\noindent\textbf{Another example}\\
Say we want to sketch the phase portrait corresponding to $\dot x=x-\cos x$, and determine the stability of all the fixed points.\\
\vspace{1mm}\\
Our approach would be to plot $x-\cos x$ and then sketch the associated vector field. This method is valid, but it requires one to figure out what the graph of $x-\cos x$ looks like. 
An easier approach would be to plot both $y=x$ and $y=\cos x$ separately; we plot both graphs on the same axes and then observe that they intersect at exactly one point:
\begin{center}
\includegraphics[width=8cm]{7}
\end{center}
This intersection corresponds to a fixed point since $x^*=\cos x^*$ and therefore $f(x^*)=0$. When the line lies above the cosine curve, we have $x>\cos x$ and so $\dot x>0$; the flow is to the
right. Similarly, the flow is to the left when the line is below the cosine curve.\\
\vspace{1mm}\\
Hence $x^*$ is the only fixed point and is unstable. See that we can classify its stability even though we don't know exactly where it is.
\newpage

\section{Linear stability analysis}
One may want to have a more quantitative measure of stability, such as the rate of decay to a stable fixed point. This sort of information
may be obtained by \textit{linearising} about a fixed point.\\
\vspace{1mm}\\
Let $x^*$ be a fixed point, and let
\begin{equation*}
x(t)=x^*+u(t)
\end{equation*}
where 
\begin{equation*}
u(t)=x(t)-x^*
\end{equation*}
differentiation yields
\begin{equation*}
\dot u=\frac{d}{dt}(x-x^*)=\dot x
\end{equation*}
Thus $\dot u=\dot x=f(x)=f(x^*+u)$. Now using Taylor's expansion we obtain
\begin{equation*}
f(x^*+u)=f(x^*)+f'(x^*)u+O(u^2)
\end{equation*}
where $O(u^2)$ denotes quadratically small terms in $u$. Finally observe that $f'(x^*)=0$ since $x^*$ is a fixed point. Hence
\begin{equation*}
\dot u=f'(x^*)u+O(u^2)
\end{equation*}
If $f'(x^*)\neq0$, the $O(u^2)$ terms are negligible in comparison to the linear term, so we may write the approximation
\begin{equation*}
\dot u\approx f'(x^*)u
\end{equation*}









\end{document}
