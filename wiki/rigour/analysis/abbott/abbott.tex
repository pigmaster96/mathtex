\documentclass{report} 
\title{Abbott}
\date{Started Jan 17 2026}
\author{Malcolm}
\usepackage{amsmath} %import math
\usepackage{mathtools} %more math
\usepackage{amssymb} %for QED symbol
\usepackage{amsthm} %
\usepackage{mathrsfs} %
\usepackage{bm}%bold math
\usepackage{graphicx} %import imaging

\graphicspath{{./images/}} %set imaging path
\newtheorem{theorem}{Theorem}
\newtheorem{lemma}{Lemma}
\theoremstyle{definition}
\newtheorem{definition}{Definition}



\begin{document}
\maketitle

\tableofcontents

\chapter{The real numbers}
\section{Definitions and the necessity for real numbers}
We begin with the \textit{natural numbers}
\begin{equation*}
\mathbb N=\{1,2,3,4,5,\ldots\}
\end{equation*}
If we restrict our attention to the natural numbers $\mathbb N$, then we can perform addition perfectly well, but if we want to have an additive identity (zero) and the additive inverses necessary
to define subtraction, we must extend our system to the \textit{integers}
\begin{equation*}
\mathbb Z=\{\ldots,-3,-2,-1,0,1,2,3,\ldots\}
\end{equation*}
The next issue is multiplication and division. The number 1 acts as the multiplicative identity, but in order to define division we need to have multiplicative inverses. Thus we 
extend our system again to the \textit{rational numbers}
\begin{equation*}
\mathbb Q=\left\{\text{all fractions $\frac{p}{q}$ where $p$ and $q$ are integers with $q\neq0$}\right\}
\end{equation*}
The properties of $\mathbb Q$ essentially make up the definition of what is called a \textit{field}. 
More formally stated, a field is any set where addition and multiplication are well-defined operations that are commutative, associative, and distributive $a(b+c)=ab+ac$. 
There must be an additive identity, and every element must have an additive inverse. 
There must be a multiplicative identity, and multiplicative inverses must exist for all nonzero elements of the field.\\
(next page)\newpage
\noindent\textbf{Cont.}\\
The set $\mathbb Q$ has a natural \textit{order} defined on it. Given any two rational numbers $r$ and $s$, exactly one of the following is true:
\begin{equation*}
r<s,\quad r=s,\quad\text{or}\quad r>s
\end{equation*}
The ordering is transitive in the sense that if $r<s$ and $s<t$ then $r<t$, so we are led to the mental picture of the rational numbers as being laid out from left to right along a number line.
Unlike $\mathbb Z$, there are no intervals of empty space; given any two rational numbers $r<s$, the rational number $(r+s)/2$ sits halfway in between, implying that the rational numbers are densely
nestled together.\\
\vspace{1mm}\\
With the field properties of $\mathbb Q$ allowing us to safely carry out addition, subtraction, multiplication, and division, we want to consider what $\mathbb Q$ is lacking.
As an example it can be proven that $\sqrt 2$ is irrational; using rational numbers it is possible 
to \textit{approximate} $\sqrt 2$ quite well:
\begin{center}
\includegraphics[width=12cm]{1}
\end{center}
For instance, $1.414^2=1.999396$, and by adding more decimal places to the approximation we can get even closer to the value for $\sqrt2$, but even so, it is clear that there is a `hole' in the 
rational number line where $\sqrt2$ ought to be. $\sqrt3$ and $\sqrt5$ are also examples of this.
If we want every length along the number line to correspond to an actual number, then another extension to our number system is in order. 
Thus, to the chain $\mathbb N\subseteq\mathbb Z\subseteq\mathbb Q$ we append the \textit{real numbers} $\mathbb R$.\\
\vspace{1mm}\\
\textbf{Intuition for real numbers}\\
The question of how to construct $\mathbb R$ from $\mathbb Q$ is complicated; it is discussed later on. For now it is not too inaccurate to say that $\mathbb R$ is obtained by 
filling in the gaps in $\mathbb Q$. Wherever there is a hole, a new \textit{irrational} number is defined and placed into the ordering that already exists on $\mathbb Q$. The real numbers
are the union of these irrational numbers together with the more familiar rational ones.
\newpage

\section{The axiom of completeness}
What exactly is a real number? We got as far as saying that the set $\mathbb R$ of real numbers is an extension of the rational numbers $\mathbb Q$ in which there are no holes or gaps, where every
length along the number line---such as $\sqrt2$---corresponds to a real number.\\
\vspace{1mm}\\
One will want to improve on this definition, and we could; continuing to use this unprecise definition would mean that whatever precise statements we formulate will rest on unproven assumptions and 
undefined terms. However, at some point we must draw a line and confess that this is what we have decided to accept as a reasonable place to start.
Naturally there is some debate as to where this line should be drawn. The majority of the material covered in these notes is attributable to many prominent mathematicians; the interesting
point is that nearly all of this work was done using intuitive assumptions about the nature of 
$\mathbb R$ quite similar to our own informal understanding at this point. It was these very theorems that motivated a rigorous construction of $\mathbb R$; not the other way around.\\
\vspace{1mm}\\
We will follow this historical model: our own rigorous construction of $\mathbb R$ from $\mathbb Q$ will be postponed till much later, when the need for such a construction will be more justified and easier to appreciate.

\subsection{An initial definition for $\mathbb R$}
$\mathbb R$ is a set containing $\mathbb Q$. The operations of addition and multiplication on $\mathbb Q$ extend to all $\mathbb R$ in such a way that every element of $\mathbb R$ has an additive 
inverse and every nonzero element of $\mathbb R$ has a multiplicative inverse. We assume $\mathbb R$ is a \textit{field}, meaning that addition and multiplication of real numbers are commutative,
associative, and distributive.
We also assume that the familiar properties of ordering on $\mathbb Q$ extend to all of $\mathbb R$ (for instance if $a<b$ and $c>0$ then $ac<bc$). More formally, we assume
that $\mathbb R$ is an \textit{ordered field}, which contains $\mathbb Q$ as a subfield (rigorous definitions left to the end).\\
\vspace{1mm}\\
This brings us to the final, and most distinctive, assumption about the real number system. We must find some way to clearly articulate what we mean by insisting that 
$\mathbb R$ does not contain the gaps that permeate $\mathbb Q$. This is the defining difference between real numbers and rational numbers, and is referred to as the 
\textit{Axiom of Completeness}:\\
\vspace{1mm}\\
\textbf{Axiom of Completeness.} \textit{Every nonempty set of real numbers that is bounded above has a least upper bound}.\\
\vspace{1mm}\\
Explaining this statement is the focus of the rest of the section.
\newpage

\subsection{Least Upper Bounds and Greatest Lower Bounds}
\begin{definition}
A set $A\subseteq\mathbb R$ is \textit{bounded above} if there exists a number $b\in\mathbb R$ such that $a\leq b$ for all $a\in A$. The number $b$ is called an \textit{upper bound} for $A$.\\
\indent Similarly, the set $A$ is \textit{bounded below} if there exists a \textit{lower bound} $l\in\mathbb R$ satisfying $l\leq a$ for every $a\in A$.
\end{definition}
\begin{definition}
A real number $s$ is the \textit{least upper bound} for a set $A\subseteq\mathbb R$ if it meets the following two criteria:
\begin{enumerate}
\item $s$ is an upper bound for $A$;
\item if $b$ is any upper bound for $A$, then $s\leq b$.
\end{enumerate}
The least upper bound is frequently called the \textit{supremum} of the set $A$. Although the notation $s=\text{lub}A$ is sometimes used, we will always write
$s=\sup A$ for the least upper bound.\\
\indent The \textit{greatest lower bound} or \textit{infimum} for $A$ is defined in a similar way. A real number $l$ is the greatest lower bound for the set $A\subseteq\mathbb R$ if
\begin{enumerate}
\item $l$ is a lower bound for $A$;
\item if $b$ is any lower bound for $A$, then $b\leq l$.
\end{enumerate}
\end{definition}
\begin{definition}
A real number $a_0$ is the \textit{maximum} of the set $A$ if $a_0$ is an element of $A$ and $a_0\geq a$ for all $a\in A$. Similarly, a number $a_1$ is a \textit{minimum} of $A$ if $a_1\in A$ and 
$a_1\leq a$ for every $a\in A$.
\end{definition}
\noindent\textbf{Axiom of Completeness.} \textit{Every nonempty set of real numbers that is bounded above has a least upper bound}.\\
\vspace{1mm}\\
The axiom of completeness asserts that every nonempty bounded set does have a least upper bound. We are not going to prove this. An \textit{axiom} in mathematics
is an accepted assumption, to be used without proof. Perferably, an axiom should be an elementary
statement about the system in question so fundamental that it seems to need no justification. 
\newpage

\subsection{Uniqueness of least upper bound}
\begin{theorem}(Antisymmetry)
$\forall x\in\mathbb R\forall y\in\mathbb R((x\leq y)\land(y\leq x)\to y=x)$
\end{theorem}
\begin{proof}(by me)
Let $x$ and $y$ be arbitrary real numbers. Suppose $x\leq y$ and $y\leq x$. Then $x-y\leq0$ and $0\leq x-y$. Then $0\leq x-y\leq0$, so $x-y=0$ and $x=y$.
\end{proof}
\begin{theorem}
If $s\in\mathbb R$ is a least upper bound of $A\subseteq\mathbb R$, then $s$ is the unique upper bound on $A$.
\end{theorem}
\begin{proof}(by me)
Suppose $s\in\mathbb R$ is a least upper bound of $A\subseteq\mathbb R$. Now suppose $x\in\mathbb R$ is also a least upper bound of $A$. Then since $s\in\mathbb R$ and $x$ is a least upper bound, 
$x\leq s$. But since $x\in\mathbb R$ and $s$ is a least upper bound, $s\leq x$. By antisymmetry, 
we have $x=s$, showing that $s$ is unique.
\end{proof}

\subsection{sup(c+A)=c+sup A}
\begin{theorem}
Let $A\subseteq\mathbb R$ be nonempty and bounded above, and let $c\in\mathbb R$. Defining the set $c+A$ by
\begin{equation*}
c+A=\{c+a\mid a\in A\}
\end{equation*}
Then $\sup(c+A)=c+\sup(A)$
\end{theorem}
\begin{proof} (by me)
Denoting $\alpha=\sup(c+A)$ and $\beta=\sup(A)$, we will show $\alpha\leq c+\beta$ and $c+\beta\leq\alpha$.\\
\indent Starting with $\alpha\leq c+\beta$, we first show that $c+\beta$ is an upper bound on $c+A$. Supposing arbitrary $a\in c+A$, we can declare some $x\in A$ such that $a=c+x$. Since $x\in A$
and $\beta=\sup(A)$, $x\leq\beta$, so $a=c+x\leq c+\beta$, and $c+\beta$ is an upper bound on $c+A$. Since $\alpha=\sup(c+A)$, $\alpha\leq c+\beta$.\\
\indent Next we show $c+\beta\leq\alpha$. We will show that $\alpha-c$ is an upper bound on $A$. Supposing arbitrary $a\in A$, then $c+a\in c+A$. Since $\alpha=\sup(c+A)$, then $c+a\leq\alpha$ and 
$a\leq\alpha-c$, so $\alpha-c$ is an upper bound on $A$. Now, since $\beta=\sup(A)$, 
$\beta\leq\alpha-c$, so $c+\beta\leq\alpha$. By the antisymmetry of real numbers, $\alpha=c+\beta$, and $\sup(c+A)=c+\sup(A)$.
\end{proof}
\noindent\textit{Commentary}\\
To prove that for real numbers $x$ and $y$, $x=y$, one strategy is to prove $x\leq y$ and $y\leq x$. This is particularly relevant here because the definition of the supremum involves inequalities.
\newpage

\subsection{}









\end{document}
