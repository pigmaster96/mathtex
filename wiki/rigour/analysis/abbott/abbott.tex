\documentclass{report} 
\title{Abbott}
\date{Started Jan 17 2026}
\author{Malcolm}
\usepackage{amsmath} %import math
\usepackage{mathtools} %more math
\usepackage{amssymb} %for QED symbol
\usepackage{amsthm} %
\usepackage{mathrsfs} %
\usepackage{bm}%bold math
\usepackage{graphicx} %import imaging

\graphicspath{{./images/}} %set imaging path
\newtheorem{theorem}{Theorem}
\newtheorem{lemma}{Lemma}
\theoremstyle{definition}
\newtheorem{definition}{Definition}



\begin{document}
\maketitle

\tableofcontents

\chapter{The real numbers}
\section{Definitions and the necessity for real numbers}
We begin with the \textit{natural numbers}
\begin{equation*}
\mathbb N=\{1,2,3,4,5,\ldots\}
\end{equation*}
If we restrict our attention to the natural numbers $\mathbb N$, then we can perform addition perfectly well, but if we want to have an additive identity (zero) and the additive inverses necessary
to define subtraction, we must extend our system to the \textit{integers}
\begin{equation*}
\mathbb Z=\{\ldots,-3,-2,-1,0,1,2,3,\ldots\}
\end{equation*}
The next issue is multiplication and division. The number 1 acts as the multiplicative identity, but in order to define division we need to have multiplicative inverses. Thus we 
extend our system again to the \textit{rational numbers}
\begin{equation*}
\mathbb Q=\left\{\text{all fractions $\frac{p}{q}$ where $p$ and $q$ are integers with $q\neq0$}\right\}
\end{equation*}
The properties of $\mathbb Q$ essentially make up the definition of what is called a \textit{field}. 
More formally stated, a field is any set where addition and multiplication are well-defined operations that are commutative, associative, and distributive $a(b+c)=ab+ac$. 
There must be an additive identity, and every element must have an additive inverse. 
There must be a multiplicative identity, and multiplicative inverses must exist for all nonzero elements of the field.\\
(next page)\newpage
\noindent\textbf{Cont.}\\
The set $\mathbb Q$ has a natural \textit{order} defined on it. Given any two rational numbers $r$ and $s$, exactly one of the following is true:
\begin{equation*}
r<s,\quad r=s,\quad\text{or}\quad r>s
\end{equation*}
The ordering is transitive in the sense that if $r<s$ and $s<t$ then $r<t$, so we are led to the mental picture of the rational numbers as being laid out from left to right along a number line.
Unlike $\mathbb Z$, there are no intervals of empty space; given any two rational numbers $r<s$, the rational number $(r+s)/2$ sits halfway in between, implying that the rational numbers are densely
nestled together.\\
\vspace{1mm}\\
With the field properties of $\mathbb Q$ allowing us to safely carry out addition, subtraction, multiplication, and division, we want to consider what $\mathbb Q$ is lacking.
As an example it can be proven that $\sqrt 2$ is irrational; using rational numbers it is possible 
to \textit{approximate} $\sqrt 2$ quite well:
\begin{center}
\includegraphics[width=12cm]{1}
\end{center}
For instance, $1.414^2=1.999396$, and by adding more decimal places to the approximation we can get even closer to the value for $\sqrt2$, but even so, it is clear that there is a `hole' in the 
rational number line where $\sqrt2$ ought to be. $\sqrt3$ and $\sqrt5$ are also examples of this.
If we want every length along the number line to correspond to an actual number, then another extension to our number system is in order. 
Thus, to the chain $\mathbb N\subseteq\mathbb Z\subseteq\mathbb Q$ we append the \textit{real numbers} $\mathbb R$.













\end{document}
