\documentclass{report} 
\title{Brualdi}
\date{Started Feb 26}
\author{Malcolm}
\usepackage{amsmath} %import math
\usepackage{mathtools} %more math
\usepackage{amssymb} %for QED symbol
\usepackage{amsthm} %
\usepackage{mathrsfs} %
\usepackage{bm}%bold math
\usepackage{graphicx} %import imaging

\graphicspath{{./images/}} %set imaging path
\newtheorem{theorem}{Theorem}
\newtheorem{lemma}{Lemma}
\theoremstyle{definition}
\newtheorem{definition}{Definition}



\begin{document}
\maketitle

\tableofcontents

\newpage
\chapter{Permutations and Combinations}
\section{Counting principles}
\begin{definition}
Let $S$ be a set. A \textit{partition} of $S$ is a collection $S_1,S_2,\ldots,S_m$ of subsets of $S$ such that each element of $S$ is 
in exactly one of those subsets:
\begin{align*}
S=S_1\cup S_2&\cup\cdots\cup S_m,\\
S_i\cap S_j=\emptyset&,\quad (i\neq j).
\end{align*}
See that the sets $S_1,S_2,\ldots,S_m$ are pairwise disjoint with union $S$. The subsets $S_1,S_2,\ldots,S_m$ are the \textit{parts} of
the partition. Note that in this definition a part of a partition may be empty, but there is usually no advantage to this. 
The \textit{number of objects} in a set $S$ is denoted $|S|$ and is sometimes called the \textit{size} of $S$.
\end{definition}
\noindent (The following are results of set theory, but are intuitive enough to be given without proofs)
\begin{theorem} (Addition principle)
Suppose that a set $S$ is partitioned into pairwise disjoint parts $S_1,S_2,\ldots,S_m$. The number of objects in $S$ can be determined
by finding the number of objects in each of the parts, and adding the numbers so obtained:
\begin{equation*}
|S|=|S_1|+|S_2|+\ldots+|S_m|
\end{equation*}
\end{theorem}





\end{document}
