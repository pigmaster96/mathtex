\documentclass{report} 
\title{Velleman}
\date{Started 6th September 2025}
\author{Malcolm}
\usepackage{amsmath} %import math
\usepackage{mathtools} %more math
\usepackage{amssymb} %for QED symbol
\usepackage{amsthm} %
\usepackage{mathrsfs} %
\usepackage{bm}%bold math
\usepackage{graphicx} %import imaging

\graphicspath{{./images/}} %set imaging path
\newtheorem*{theorem}{Theorem}
\theoremstyle{definition}
\newtheorem*{definition}{Definition}


\begin{document}
\maketitle

\tableofcontents

\newpage
\chapter{Logic}
\subsection{Logic Factsheet}
\textbf{De Morgan's laws}
\begin{align*}
&\neg\,(P\land Q)\text{ is equivalent to }\neg\,P\lor\neg\,Q\\
&\neg\,(P\lor Q)\text{ is equivalent to }\neg\,P\land\neg\,Q
\end{align*}
\textbf{Commutative laws}
\begin{align*}
P\land Q\text{ is equivalent to }Q\land P\\
P\lor Q\text{ is equivalent to }Q\lor P
\end{align*}
\textbf{Associative laws}
\begin{align*}
P\land(Q\land R)\text{ is equivalent to }(P\land Q)\land R\\
P\lor(Q\lor R)\text{ is equivalent to }(P\lor Q)\lor R
\end{align*}
\textbf{Indempotent laws}
\begin{align*}
P\land P\text{ is equivalent to }P\\
P\lor P\text{ is equivalent to }P
\end{align*}
\textbf{Distributive laws}
\begin{align*}
P\land(Q\lor R)\text{ is equivalent to }(P\land Q)\lor(P\land R)\\
P\lor(Q\land R)\text{ is equivalent to }(P\lor Q)\land(P\lor R)
\end{align*}
\textbf{Absorption laws}
\begin{align*}
P\lor(P\land Q)\text{ is equivalent to }P\\
P\land(P\lor Q)\text{ is equivalent to }P
\end{align*}
\textbf{Double Negation law}
\begin{align*}
\neg\neg\,P\text{ is equivalent to }P
\end{align*}
\newpage

\subsection{Set operation definitions}
The \textit{intersection} of two sets $A$ and $B$ is the set $A\cap B$ defined as follows:
\begin{align*}
A\cap B=\{x\,|\,x\in A\text{ and }x\in B\}
\end{align*}
The \textit{union} of $A$ and $B$ is the set $A\cup B$ defined as follows:
\begin{align*}
A\cup B=\{x\,|\,x\in A\text{ or }x\in B\}
\end{align*}
The \textit{difference} of $A$ and $B$ is the set $A\setminus B$ defined as follows:
\begin{align*}
A\setminus B=\{x\,|\,x\in A\text{ and }x\notin B\}
\end{align*}
See that 
\begin{equation*}
x\in A\cap B=x\in\{y\,|\,y\in A\text{ and }y\in B\}
\end{equation*}
where $y$ is a dummy variable. So we can also write that
\begin{equation*}
x\in A\cap B=x\in A\land x\in B
\end{equation*}
The same can be shown for the union and difference.

\subsection{Distributivity of set operations}
We show \begin{equation*} x\in A\cap(B\cup C)\text{ is equivalent to }
x\in(A\cap B)\cup(A\cap C)
\end{equation*}
By analysing their logical forms:
\begin{align*}
&x\in A\cap(B\cup C)\\
&=x\in A\land x\in(B\cup C)\\
&=x\in A\land(x\in B\lor x\in C)
\end{align*}
and
\begin{align*}
&x\in(A\cap B)\cup(A\cap C)\\
&=x\in(A\cap B)\lor x\in(A\cap C)\\
&=(x\in A\land x\in B)\lor(x\in A\land x\in C)\\
&=[(x\in A\land x\in B)\lor x\in A)]\land[(x\in A\land x\in B)\lor x\in C)]\\
&=x\in A\land[(x\in A\lor x\in C)\land(x\in B\lor x\in C)]\\
&=[x\in A\land(x\in A\lor x\in C)]\land(x\in B\lor x\in C)\\
&=x\in A\land(x\in B\lor x\in C)
\end{align*}
We can also show, in a similar manner, that
\begin{equation*}
x\in A\cup(B\cap C)\text{ is equivalent to }
x\in(A\cup B)\cap(A\cup C)
\end{equation*}
\newpage

\subsection{$x\in A\setminus(B\cap C)=x\in(A\setminus B)\cup(A\setminus C)$}
We can also show
\begin{equation*}
x\in A\setminus(B\cap C)=x\in(A\setminus B)\cup(A\setminus C)
\end{equation*}
See that
\begin{align*}
&x\in A\setminus(B\cap C)&\\
&=x\in A\land\neg\,(x\in B\cap C)&\text{(Definition of $\setminus$)}\\
&=x\in A\land\neg\,(x\in B\land x\in C)&\text{(Definition of $\cap$)}\\
&=x\in A\land(x\notin B\lor x\notin C)&\text{(De Morgan's)}\\
&=(x\in A\land x\notin B)\lor(x\in A\land x\notin C)&\text{(Distributivity)}\\
&=(x\in A\setminus B)\lor(x\in A\setminus C)&\text{(Definition of $\setminus$)}\\
&=x\in(A\setminus B)\cup(A\setminus C)&\text{(Definition of $\cup$)}
\end{align*}

\subsection{$x\in(A\cup B)\setminus(A\cap B)=x\in(A\setminus B)\cup(B\setminus A)$}
\begin{align*}
&x\in(A\cup B)\setminus(A\cap B)&\\
&=(x\in A\lor x\in B)\land\neg\,(x\in A\land x\in B)&(\text{By definition)}\\
&=(x\in A\lor x\in B)\land(x\notin A\lor x\notin B)&\text{(De Morgan's)}\\
&=[(x\in A\lor x\in B)\land(x\notin A)]&\\
&\qquad\qquad\qquad\lor[(x\in A\lor x\in B)\land(x\notin B)]&\text{(Distributivity)}\\
&=[(x\notin A\land x\in A)\lor(x\notin A\land x\in B)]&\\
&\qquad\qquad\qquad\lor[(x\notin B\land x\in A)\lor(x\notin B\land x\in B)]&\text{(Distributivity)}\\
&=(x\notin A\land x\in B)\lor(x\notin B\land x\in A)&\\
&=(x\in A\land x\notin B)\land(x\in B\land x\notin A)&\text{(Commutativity)}\\
&=x\in(A\setminus B)\cup(B\setminus A)&\text{(By definition)}
\end{align*}

\subsection{$(A\cap B)\cap(A\setminus B)=\emptyset$}
See that
\begin{align*}
&x\in(A\cap B)\cap(A\setminus B)&\\
&=(x\in A\land x\in B)\land(x\in A\land x\notin B)&\text{(Definition)}\\
&=x\in A\land\underbrace{(x\in B\land x\notin B)}_{\text{Contradiction}}&\text{(Associativity + Commutativity)}
\end{align*}
The last statement is a contradiction, so the statement 
$x\in(A\cap B)\cap(A\setminus B)$ will always be false, no matter what $x$ is. In other words, nothing can be an element of 
$(A\cap B)\cap(A\setminus B)$, so it must be the case that 
$(A\cap B)\cap(A\setminus B)=\emptyset$; $A\cap B$ and $A\setminus B$ are disjoint.
\newpage

\subsection{Conditional and Contrapositive laws}
\textbf{Conditional Law}
\begin{equation*}
P\to Q\text{ is equivalent to }\neg\,(P\land\neg\,Q)
\end{equation*}
by De Morgan's law we can also say that
\begin{equation*}
P\to Q\text{ is equivalent to }\neg\,P\lor Q
\end{equation*}
\textbf{Contrapositive law}
\begin{equation*}
P\to Q\text{ is equivalent to }\neg\,Q\to\neg\,P
\end{equation*}
This can be justified using
\begin{equation*}
P\to Q=\neg\,(P\land\neg\,Q)=\neg\,(\neg\,Q\land P)=\neg\,Q\to\neg\,P
\end{equation*}
\textbf{Intuition}\\
Intuitive ways to think of $P\to Q$ (and equivalently $\neg\,Q\to\neg\,P$) include:
\begin{itemize}
\item $P\text{ implies }Q$.
\item $Q\text{, if }P$.
\item $P\text{ only if }Q$.
\item $P\text{ is a sufficient condition for }Q$.
\item $Q\text{ is a necessary condition for }P$.
\end{itemize}
\newpage

\subsection{Biconditional statements}
We write
\begin{equation*}
P\leftrightarrow Q=(P\to Q)\land(Q\to P)
\end{equation*}
Note that by the contrapositive law, this is also equivalent to
\begin{equation*}
(P\to Q)\land(\neg\,P\to\neg\,Q)
\end{equation*}
\textbf{Intuition}\\
$Q\to P$ can be written as `$P$ if $Q$' and $P\to Q$ can be written as `$P$ only if $Q$' (since this means $\neg\,Q\to\neg\,P$ which is $P\to Q$).\\
\vspace{1mm}\\
Combining the two as $(P\to Q)\land(Q\to P)=P\leftrightarrow Q$ therefore corresponds to the statement 
`$P$ if and only if $Q$'.\\
\vspace{1mm}\\
$P\leftrightarrow Q$ means `$P$ iff $Q$', or `$P$ is a necessary and sufficient condition for $Q$'.
\newpage

\section{Quantificational logic}
\subsection{Quantifier negation laws}
We have
\begin{align*}
&\neg\exists xP(x)\text{ is equivalent to }\forall x\neg P(x)\\
&\neg\forall xP(x)\text{ is equivalent to }\exists x\neg P(x)
\end{align*}
\textbf{Intuition}\\
No matter what $P(x)$ stands for, the formula $\neg\exists xP(x)$ means that there is no value of $x$ for which $P(x)$ is true; this is ths same as saying that for every value of $x$ in 
the universe of discourse, $P(x)$ is false---meaning $\forall x\neg P(x)$.\\
\vspace{1mm}\\
Similarly, to say that $\neg\forall xP(x)$ means that it is not the case that for all values of $x$, $P(x)$ is true. This is equivalent to saying that that there is at least one
value of $x$ for which $P(x)$ is false---so $\exists x\neg P(x)$.

\subsection{Notation}
\textbf{`Exactly one' notation}\\
We write 
\begin{equation*}
\exists!xP(x)=\exists x(P(x)\land\neg\exists y(P(y)\land y\neq x))
\end{equation*}
As a shorthand way to write `there is exactly one value of $x$ such that $P(x)$ is true', or `there is a unique $x$ such that $P(x)$'.\\
\vspace{1mm}\\
\textbf{Specifying quantifiers}\\
We write
\begin{equation*}
\forall x\in A\, P(x)
\end{equation*}
to mean that \textit{for every value of $x$ in the set $A$, $P(x)$ is true}. Similarly,
\begin{equation*}
\exists x\in A\, P(x)
\end{equation*}
means \textit{there is at least one value of $x $ in the set $A$ such that $P(x)$ is true}.\\
\vspace{1mm}\\
Formulas containing bounded quantifiers can also be thought of as abbreviations for more complicated formulas containing only normal, unbounded quantifiers. See that 
\begin{equation*}
\forall x\in A\, P(x)=\forall x(x\in A\to P(x))
\end{equation*}
and
\begin{equation*}
\exists x\in A\,P(x)=\exists x(x\in A\land P(x))
\end{equation*}
\newpage

\subsection{Negation law for bounded quantifiers}
We can show
\begin{equation*}
\neg\forall x\in A\,P(x)=\exists x\in A\,\neg P(x)
\end{equation*}
See that
\begin{align*}
&\neg\forall x\in A\,P(x)&\\
&=\neg\forall x(x\in A\to P(x))&\text{(as defined)}\\
&=\exists x\neg(x\in A\to P(x))&\text{(negation law)}\\
&=\exists x\neg\neg(x\in A\land\neg P(x))&\text{(conditional law)}\\
&=\exists x(x\in A\land\neg P(x))&\\
&=\exists x\in A\,\neg P(x)&\text{(as defined)}
\end{align*}
Similarly we can show
\begin{equation*}
\neg\exists x\in A\,P(x)=\forall x\in A\,\neg P(x)
\end{equation*}
See that
\begin{align*}
&\neg\exists x\in A\,P(x)&\\
&=\neg\exists x(x\in A\land P(x))&\text{(as defined)}\\
&=\forall x\neg(x\in A\land P(x))&\text{(negation law)}\\
&=\forall x(x\in A\to\neg P(x))&\text{(conditional law)}\\
&=\forall x\in A\,\neg P(x)&\text{(as defined)}
\end{align*}
\newpage

\subsection{Vacuously true}
It is clear that if $A=\emptyset$ then $\exists x\in A\,P(x)$ will be false regardless of $P(x)$, since there is nothing in $A$ that makes $P(x)$ come true 
(since there is nothing in $A$ to being with).\\
\vspace{1mm}\\
Now consider $\forall x\in A\,P(x)$. We can reason that
\begin{equation*}
\forall x\in A\,P(x)=\neg\exists x\in A\,\neg P(x)\quad\text{(quantifier negation)}
\end{equation*}
See that if $A=\emptyset$ then this formula will be true, no matter what $P(x)$ is. In this case we say that the statement is \textit{vacuously true}.\\
\vspace{1mm}\\
Another way to see this is to rewrite
\begin{equation*}
\forall x\in A\,P(x)=\forall x(x\in A\to P(x))
\end{equation*}
The only way this can be false is if there is some value of $x$ such that $x\in A$ is true but $P(x)$ false; but there is no such value of $x$.
Intuitively, because the condition cannot be met, it is impossible to provide a counterexample to prove something wrong.\\
\vspace{1mm}\\
An analogy would be me claiming `i've never lost a race to Usain Bolt'. This is true, but vacuously so.
\newpage

\subsection{Alternate definition for indexed families}
Say we are looking for the set $\{p_1,p_2,\ldots,p_{100}\}$; another way of describing this set would be to say that it consists of all numbers $p_i$, for $i$ an element of the set
$I=\{1,2,3,\ldots,100\}=\{i\in\mathbb{N}|1\leq i\leq 100\}$. We can write
\begin{equation*}
P=\{p_i|i\in I\}
\end{equation*}
Each element $p_i$ in this set is identified by $i\in I$, called the \textit{index} of each element. A set defined this way is called an \textit{indexed family}, and $I$ the \textit{index set}.
Although the indices for an indexed family are often numbers, they need not be.\\
\vspace{1mm}\\
In general, see that any indexed family
\begin{equation*}
A=\{x_i|i\in I\}
\end{equation*}
Can also be defined as
\begin{equation*}
A=\{x|\exists i\in I\,(x=x_i)\}
\end{equation*}
It follows that the statement
\begin{equation*}
x\in\{x_i|i\in I\}
\end{equation*}
means the same thing as
\begin{equation*}
\exists i\in I\,(x=x_i)
\end{equation*}
\newpage

\subsection{Power set}
Suppose $A$ is as set. The \textit{power set} of $A$, denoted $\mathscr{P}(A)$, is the set whose elements are all subsets of $A$. In other words,
\begin{equation*}
\mathscr{P}(A)=\{x|x\subseteq A\}
\end{equation*}
For instance, the set $A=\{7,12\}$ has four subsets $\emptyset,\{7\},\{12\}$, and $\{7,12\}$;
thus, $\mathscr{P}(A)=\{\emptyset,\{7\},\{12\},\{7,12\}\}$. 

\subsection{Intersection and union of a family of sets}
Suppose $\mathcal{F}$ is a family of sets. The \textit{intersection} and \textit{union} of $\mathcal{F}$ are the sets $\bigcap\mathcal{F}$ and $\bigcup\mathcal{F}$ are defined as follows:
\begin{align*}
&\bigcap\mathcal F=\{x|\forall A\in\mathcal F(x\in A)\}=\{x|\forall A(A\in\mathcal F\to x\in A)\}\\
&\bigcup\mathcal F=\{x|\exists A\in\mathcal F(x\in A)\}=\{x|\exists A(A\in\mathcal F\land x\in A)\}
\end{align*}
Notice that if $A$ and $B$ are any two sets and $\mathcal F=\{A,B\}$, then $\bigcap\mathcal F=A\cap B$ and $\bigcup\mathcal F=A\cup B$; the definitions of intersection and union of a family of sets
are generalisations of our old definitions of the intersection and union of two sets.\\
\vspace{1mm}\\
\textbf{Alternative notation}\\
An alternative notation is sometimes used for the union or intersection of an indexed family of sets. Suppose $\mathcal F=\{A_i|i\in I\}$, where each $A_i$ is a set,
then $\bigcap\mathcal F$ and $\bigcup\mathcal F$ could also be written as 
$\bigcap_{i\in I}A_i$ and $\bigcup_{i\in I}A_i$; as such
\begin{align*}
&\bigcap\mathcal F=\bigcap_{i\in I}A_i=\{x|\forall i\in I(x\in A_i)\}\\
&\bigcup\mathcal F=\bigcup_{i\in I}A_i=\{x|\exists i\in I(x\in A_i)\}
\end{align*}
\newpage

\subsection{More on set notation}
One generally defines a set using the elementhood test notation
\begin{equation*}
\{x|P(x)\}
\end{equation*}
Where the set consists of all $x$ that satisfy the specified condition $P(x)$. Sometimes this notation can be modified to allow the $x$ before the vertical line to be replaced
with a more complex expression. For example, suppose we wanted to define $S$ to be the set of all
perfect squres, we could write
\begin{equation*}
S=\{n^2|n\in\mathbb{N}\}
\end{equation*}
This is the same as
\begin{equation*}
S=\{x|\exists n\in\mathbb N(x=n^2)\}
\end{equation*}
See therefore that 
\begin{equation*}
x\in\{n^2|n\in\mathbb{N}\}=\exists n\in\mathbb N(x=n^2)
\end{equation*}

\chapter{Proof Strategies}
\subsection{Terminology}
We want to state the answer to a mathematical question in the form of a \textit{theorem} that says that if certain assumptions called the \textit{hypotheses} of the theorem are true, then
some conclusion must also be true.\\
\vspace{1mm}\\
An assignment of particular values to these variables is called an \textit{instance} of the theorem, and in order for the theorem to be correct it must be the case
that for every instance of the theorem that makes the hypotheses come out true, the conclusion is also true.\\
\vspace{1mm}\\
If there is even one instance
in which the hypotheses are true but the conclusion is false, then the theorem is incorrect; such an instance is called a \textit{counterexample} to the theorem.\\
\vspace{1mm}\\
As in the next section, we will refer to statements that are known or assumed to be true at some point in the course of figuring out the proof as \textit{givens}, and the statements that remains
to be proven at that point as the \textit{goal}.
\newpage

\section{To prove a conclusion of the form $P\to Q$}
To prove a conclusion of the form $P\to Q$, we can \textit{assume $P$ is true and then prove $Q$}.\\
\vspace{1mm}\\
Assuming that $P$ is true amounts to adding $P$ to the lists of hypotheses. If the conclusion of the theorem we are trying to prove has the form $P\to Q$, then we can \textit{transform the
problem} by adding $P$ to the list of hypotheses and changing the conclusion form $P\to Q$ to $Q$.\\
\vspace{1mm}\\
How we solve this new problem will now then be guided by the logical form of the new conclusion $Q$, and perhaps also that of the new hypothesis $P$.\\
\vspace{1mm}\\
This strategy is one that proves the \textit{goal} of $P\to Q$. Even if the conclusion of a theorem is not a conditional statement, if we transform the problem in such a way that the conditional
statement becomes the goal, then we can apply this strategy as the next step in figuring out the proof.\\
\vspace{1mm}\\
\textbf{Example:}
\begin{theorem}
Suppose $a$ and $b$ are real numbers. If $0<a<b$ then $a^2<b^2$.
\end{theorem}
\begin{proof}
Suppose $0<a<b$. Multiplying the inequality $a<b$ by the positive number $a$ we can conclude that $a^2<ab$; similarly multiplying by $b$ we get $ab<b^2$. Therefore $a^2<ab<b^2$, so $a^2<b^2$. Thus, if $0<a<b$ then $a^2<b^2$.
\end{proof}
\noindent We were given as a hypothesis that $a$ and $b$ are real numbers with a conclusion of the form $P\to Q$, where $P$ is the statement $0<a<b$ and $Q$ the statement $a^2<b^2$. Thus
we start with these statements as given and goal:
\begin{center}
\begin{tabular}{c|c}
\textit{Givens}&\textit{Goal}\\
$a$ and $b$ are real numbers&$(0<a<b)\to(a^2<b^2)$
\end{tabular}
\end{center}
As per our strategy, we assume that $0<a<b$ and try to use this assumption to prove $a^2<b^2$. In other words we add $0<a<b$ to the list of givenas and make $a^2<b^2$ our goal:
\begin{center}
\begin{tabular}{c|c}
\textit{Givens}&\textit{Goal}\\
$a$ and $b$ are real numbers&$a^2<b^2$\\
$0<a<b$&
\end{tabular}
\end{center}
(next page)\newpage
\noindent\textbf{Generalising}\\
Here's a restatement of the proof strategy we discussed, in the form we will be using to present proof strategies from now on.\\
\vspace{1mm}\\
\indent To prove a goal of the form $P\to Q$:\\
\indent Assume $P$ is true and then prove $Q$.\\
\vspace{1mm}\\
\textit{Scratch work}\\
Before using strategy:
\begin{center}
\begin{tabular}{c|c}
\textit{Givens}&\textit{Goal}\\
\hline
---&$P\to Q$\\
---&
\end{tabular}
\end{center}
After using strategy:
\begin{center}
\begin{tabular}{c|c}
\textit{Givens}&\textit{Goal}\\
\hline
---&$Q$\\
---&\\
$P$&
\end{tabular}
\end{center}
\textit{Form of final proof:}\\
\indent Suppose $P$.\\
\indent\indent [Proof of $Q$ goes here.]\\
\indent Therefore $P\to Q$.\\
\vspace{1mm}\\
(next page)\newpage

\subsection{Alternative approach}
Another approach could utilise the contrapositive law, where 
\begin{equation*}
P\to Q=\neg\,Q\to\neg\,P
\end{equation*}
In other words:\\
\vspace{1mm}\\
\indent To prove a goal of the form $P\to Q$:\\
\indent Assume $Q$ is false and prove that $P$ is false.\\
\vspace{1mm}\\
\textit{Scratch work}\\
Before using strategy:
\begin{center}
\begin{tabular}{c|c}
\textit{Givens}&\textit{Goal}\\
\hline
---&$P\to Q$\\
---&
\end{tabular}
\end{center}
After using strategy:
\begin{center}
\begin{tabular}{c|c}
\textit{Givens}&\textit{Goal}\\
\hline
---&$\neg\,P$\\
---&\\
$\neg\,Q$&
\end{tabular}
\end{center}
\textit{Form of final proof:}\\
\indent Suppose $Q$ is false.\\
\indent\indent [Proof of $\neg\,P$ goes here.]\\
\indent Therefore $P\to Q$.\\
\vspace{1mm}\\
(next page)\newpage
\noindent\textbf{Example}\\
Suppose $a,b$ and $c$ are real numbers and $a>b$. Prove that if $ac\leq bc$ then $c\leq0$.\\
\vspace{1mm}\\
\textit{Scratch work}\\
Before using strategy:
\begin{center}
\begin{tabular}{c|c}
\textit{Givens}&\textit{Goal}\\
\hline
$a,b$ and $c$ are real numbers&$(ac\leq bc)\to(c\leq0)$\\
$a>b$&
\end{tabular}
\end{center}
The contrapositive of the goal is $\neg(c\leq0)\to\neg(ac\leq bc)$, or $(c>0)\to(ac>bc)$. As per the previous strategy, we can prove this by adding $c>0$ to the list of givens 
and making $ac>bc$ our new goal:
\begin{center}
\begin{tabular}{c|c}
\textit{Givens}&\textit{Goal}\\
\hline
$a,b$ and $c$ are real numbers&$ac>bc$\\
$a>b$&\\
$c>0$&
\end{tabular}
\end{center}
\textit{Form of final proof}:\\
\indent Suppose $c>0$.\\
\indent\indent [Proof of $ac>bc$ goes here.]\\
\indent Therefore, if $ac\leq bc$ then $c\leq0$.\\
\vspace{1mm}\\
\textit{Solution}
\begin{theorem}
Suppose $a,b$ and $c$ are real numbers and $a>b$. If $ac\leq bc$ then $c\leq0$.
\end{theorem}
\begin{proof}
We will prove the contrapositive. Suppose $c>0$. Then we can multiply both sides of the given inequality $a>b$ by $c$ and conclude that $ac>bc$. Therefore if $ac\leq bc$ then $c\leq0$.
\end{proof}
\newpage

\section{Proofs involving negations and conditionals}
\subsection{Reexpress}
We now consider proofs in which the goal has the form $\neg P$. Usually it's easier to prove a positive statement than a negative statement, so it is often helpful to reexpress the goal before
proving it. Thus our first strategy for proving negated statements is:\\
\vspace{1mm}\\
\indent To prove a goal of the form $\neg P$:\\
\indent If possible, reexpress the goal in some other form.\\
\vspace{1mm}\\
\textbf{Example}\\
Suppose $A\cap C\subseteq B$ and $a\in C$. Prove that $a\notin A\setminus B$.\\
\vspace{1mm}\\
\textit{Scratch Work}
\begin{center}
\begin{tabular}{c|c}
\textit{Givens}&\textit{Goal}\\
\hline
$A\cap C\subseteq B$&$a\notin A\setminus B$\\
$a\in C$&
\end{tabular}
\end{center}
We have a negative goal, which we can try to reexpress as a positive statement:
\begin{align*}
a\notin A\setminus B&=\neg(a\in A\land a\notin B)&\text{(Definition)}\\
&=a\in A\to a\in B&\text{(Conditional law)}
\end{align*}
This gives us
\begin{center}
\begin{tabular}{c|c}
\textit{Givens}&\textit{Goal}\\
\hline
$A\cap C\subseteq B$&$a\in A\to a\in B$\\
$a\in C$&
\end{tabular}
\end{center}
We can apply the strategy for conditional goals:
\begin{center}
\begin{tabular}{c|c}
\textit{Givens}&\textit{Goal}\\
\hline
$A\cap C\subseteq B$&$a\in B$\\
$a\in C$&\\
$a\in A$&
\end{tabular}
\end{center}
See how the proof is now much more straightforward; from the givens $a\in A$ and $a\in C$ we can conclude $a\in A\cap C$. Since $A\cap C\subseteq B$, it follows that $a\in B$.\\
\vspace{1mm}\\
\textit{Solution}
\begin{theorem} 
Suppose $A\cap C\subseteq B$ and $a\in C$. Then $a\notin A\setminus B$.
\end{theorem}
\begin{proof} 
Suppose $a\in A$. Then since $a\in C$, $a\in A\cap C$. Since $A\cap C\subseteq B$ it follows that $a\in B$. Thus it cannot be the case that $a$ is an element of $A$ but not $B$, so 
$a\notin A\setminus B$.
\end{proof}
\newpage

\subsection{Proof by contradiction}
Say a goal of the form $\neg P$ cannot be reexpressed as a positive statement, in this case one could attempt \textit{proof by contradiction}---start by assuming $P$ is true, and try to use this
assumption to prove that something one already knows is false.\\
\vspace{1mm}\\
Often this is done by proving a statement that contradicts one of the givens; because one knows that the statement proven is false, the assumption that $P$ was true must have been incorrect---the
only remaining possibility then is that $P$ is false.\\
\vspace{1mm}\\
\indent To prove a goal of the form $\neg P$:\\
\indent Assume $P$ is true and try to reach a contradiction. In which case $P$ must be false.\\
\vspace{1mm}\\
\textit{Scratch Work}
\begin{center}
\begin{tabular}{c|c}
\textit{Givens}&\textit{Goal}\\
\hline
---&$\neg P$\\
---&
\end{tabular}
\end{center}
After using strategy:
\begin{center}
\begin{tabular}{c|c}
\textit{Givens}&\textit{Goal}\\
\hline
---&Contradiction\\
---&\\
$P$&
\end{tabular}
\end{center}
\textit{Form of final proof}:\\
\indent Suppose $P$ is true.\\
\indent\indent [Proof of contradiction goes here.]\\
\indent Thus, $P$ is false.\\
\vspace{1mm}\\
(next page)\newpage
\noindent\textbf{Example}\\
Prove that if $x^2+y=13$ and $y\neq4$ then $x\neq3$.\\
\vspace{1mm}\\
\textit{Scratch work}\\
The goal is a conditional statement. We treat the antecedent as a given and make the consequent our new goal:
\begin{center}
\begin{tabular}{c|c}
\textit{Givens}&\textit{Goal}\\
\hline
$x^2+y=13$&$x\neq3$\\
$y\neq4$&
\end{tabular}
\end{center}
Our current idea of the proof structure looks like\\
\vspace{1mm}\\
\indent Suppose $x^2+y^2=13$ and $y\neq4$.\\
\indent\indent [Proof of $x\neq3$ goes here.]\\
\indent Thus, if $x^2+y^2=13$ and $y\neq4$ then $x\neq3$.\\
\vspace{1mm}\\
In this sense, each manipulation of our problem dictates the structure of our final proof. At this point the first and last sentences of the final proof have been produced. 
What remains is to prove $x\neq 3$ given our manipulated problem.\\
\vspace{1mm}\\
The goal $x\neq3$ means $\neg(x=3)$; we try proof by contradiction and transform the problem as follows:
\begin{center}
\begin{tabular}{c|c}
\textit{Givens}&\textit{Goal}\\
\hline
$x^2+y=13$&Contradiction\\
$y\neq4$&\\
$x=3$&
\end{tabular}
\end{center}
Once again, the proof strategy that suggested this transformation also tells us how to fill in a few more sentences of the final proof:\\
\vspace{1mm}\\
\indent Suppose $x^2+y^2=13$ and $y\neq4$.\\
\indent\indent Suppose $x=3$\\
\indent\indent\indent [Proof of contradiction goes here.]\\
\indent\indent Therefore $x\neq3$\\
\indent Thus, if $x^2+y^2=13$ and $y\neq4$ then $x\neq3$.\\
\vspace{1mm}\\
The first and last lines go together and indicate that we are proving a conditional statement by assuming the antecedent and proving the consequent. Between these lines is a proof of
the consequent $x\neq3$. This inner proof has the form of a proof by contradiction, as indicated by the second first and second last lines; between these lines we still need
to fill in a proof of a contradiction.\\
\vspace{1mm}\\
At this point we don't have a particular statement as a goal; any impossible conclusion will do. We must look closely at the givens to find a contradiction.\\ 
(next page)\newpage
\noindent\textbf{Example cont.}\\
\textit{Solution}
\begin{theorem}
If $x^2+y=13$ and $y\neq4$ then $x\neq3$.
\end{theorem}
\begin{proof}
Suppose $x^2+y=13$ and $y\neq4$. Suppose $x=3$. Substituting this into the equation $x^2+y=13$, we get $9+y=13$, so $y=4$. But this contradicts the fact that $y\neq4$. Therefore $x\neq 3$. 
Thus, if $x^2+y=13$ and $y\neq 4$ then $x\neq 3$.
\end{proof}

\subsection{To use a given of the form $\neg P$ in proof by contradiction}
If attempting a proof by contradiction with a given $\neg P$. Here is a useful strategy;\\
\vspace{1mm}\\
\indent To use a given of the form $\neg P$:\\
\indent Try making $P$ the goal. If one can prove $P$, then the proof is complete, since $P$ contradicts the given $\neg P$.\\
\vspace{1mm}\\
\textit{Scratch Work}\\
Before using strategy:
\begin{center}
\begin{tabular}{c|c}
\textit{Givens}&\textit{Goal}\\
\hline
$\neg P$&Contradiction\\
---&\\
---&
\end{tabular}
\end{center}
After using strategy:
\begin{center}
\begin{tabular}{c|c}
\textit{Givens}&\textit{Goal}\\
\hline
$\neg P$&$P$\\
---&\\
---&
\end{tabular}
\end{center}
\textit{Form of final proof:}\\
\indent\indent [Proof of $P$ goes here.]\\
\indent Since we already know $\neg P$, this is a contradiction.\\
\vspace{1mm}\\
Note that proof by contradiction is not restricted to goals of the form $\neg P$, and can be used for any goal.\\
(next page)\newpage
\noindent\textbf{Example}\\
Suppose $A,B$ and $C$ are sets, $A\setminus B\subseteq C$, and $x$ is anything at all. Prove that if $x\in A\setminus C$ then $x\in B$.\\
\vspace{1mm}\\
\textit{Scratch work}\\
We're given $A\setminus B\subseteq C$ and our goal is $x\in A\setminus C\to x\in B$. We use the previously discussed strategy for conditionals (assume antecedent and prove consequent):
\begin{center}
\begin{tabular}{c|c}
\textit{Givens}&\textit{Goal}\\
\hline
$A\setminus B\subseteq C$&$x\in B$\\
$x\in A\setminus C$&
\end{tabular}
\end{center}
Our proof currently looks like\\
\indent Suppose $x\in A\setminus C$.\\
\indent\indent [Proof of $x\in B$ goes here.]\\
\indent Thus, if $x\in A\setminus C$ then $x\in B$.\\
\vspace{1mm}\\
We attempt proof by contradiction:
\begin{center}
\begin{tabular}{c|c}
\textit{Givens}&\textit{Goal}\\
\hline
$A\setminus B\subseteq C$&Contradiction\\
$x\in A\setminus C$&\\
$x\notin B$&
\end{tabular}
\end{center}
Now our proof looks like\\
\indent Suppose $x\in A\setminus C$.\\
\indent\indent Suppose $x\notin B$\\
\indent\indent\indent [Proof of contradiction goes here.]\\
\indent\indent Therefore $x\in B$\\
\indent Thus, if $x\in A\setminus C$ then $x\in B$.\\
\vspace{1mm}\\
Because we're doing a proof by contradiction and our last given is now a negated statement, we could try our strategy for using givens of the form $\neg P$. Unfortunately, this strategy suggests
making $x\in B$ our goal which just gets us back to where we started. We must look at the other givens to try to find the contradiction.\\
(next page)\newpage
\noindent\textbf{Example cont.}\\
We have
\begin{center}
\begin{tabular}{c|c}
\textit{Givens}&\textit{Goal}\\
\hline
$A\setminus B\subseteq C$&Contradiction\\
$x\in A\setminus C$&\\
$x\notin B$&
\end{tabular}
\end{center}
In this case writing out the definition of the second given is key to the proof. By definition $x\in A\setminus C$ means $x\in A$ and $x\notin C$. Replacing this given
by its definition gives us
\begin{center}
\begin{tabular}{c|c}
\textit{Givens}&\textit{Goal}\\
\hline
$A\setminus B\subseteq C$&Contradiction\\
$x\in A$&\\
$x\notin C$&\\
$x\notin B$&
\end{tabular}
\end{center}
Now the third given has the form $\neg P$. We can apply the strategy for using givens of the form $\neg P$ and make $x\in C$ our goal, where showing it would complete the proof by contradicting
the given $x\notin C$:
\begin{center}
\begin{tabular}{c|c}
\textit{Givens}&\textit{Goal}\\
\hline
$A\setminus B\subseteq C$&$x\in C$\\
$x\in A$&\\
$x\notin C$&\\
$x\notin B$&
\end{tabular}
\end{center}
Once again, we can add a little more to the proof we are gradually forming with each step. We add in the fact that we plan to derive our contradiction by proving $x\in C$; we
also add the definition of $x\in A\setminus C$ to the proof.\\
\vspace{1mm}\\
\indent Suppose $x\in A\setminus C$. This means $x\in A$ and $x\notin C$.\\
\indent\indent Suppose $x\notin B$\\
\indent\indent\indent\indent [Proof of $x\in C$ goes here.]\\
\indent\indent\indent This contradicts the fact that $x\notin C$.\\
\indent\indent Therefore $x\in B$\\
\indent Thus, if $x\in A\setminus C$ then $x\in B$.\\
\vspace{1mm}\\
We have finally reached a point where the goal follows apparently from the givens---from $x\in A$ and $x\notin B$ we conclude that $x\in A\setminus B$. Since $A\setminus B\subseteq C$ it
follows that $x\in C$.\\
(next page)\newpage
\noindent\textbf{Example cont.}\\
\textit{Solution}
\begin{theorem}
Suppose $A,B$ and $C$ are sets, $A\setminus B\subseteq C$, and $x$ is anything at all. If $x\in A\setminus C$ then $x\in B$.
\end{theorem}
\begin{proof}
Suppose $x\in A\setminus C$. This means that $x\in A$ and $x\notin C$. Suppose $x\notin B$. Then
$x\in A\setminus B$, so since $A\setminus B\subseteq C$, $x\in C$. But this contradicts the fact that $x\notin C$. Therefore $x\in B$. Thus, if $x\in A\setminus C$ then $x\in B$.
\end{proof}

\subsection{Given conditionals}
We consider strategies for using givens of the form $P\to Q$:\\
\vspace{1mm}\\
\indent To use a given of the form $P\to Q$:\\
\indent If also given $P$, or if one can prove $P$, then one can use this to conclude $Q$. Another approach would be to use its equivalence to $\neg Q\to\neg P$; if one can prove that
$Q$ is false, then one can conclude that $P$ is false.\\
\vspace{1mm}\\
The first rule says that if you know both $P$ and $P\to Q$ are true, then $Q$ must also be true; this rule is called \textit{modus ponens}. The second rule, called \textit{modus tollens}, 
says that if you know $P\to Q$ is true and $Q$ is false, then you can conclude that $P$ must also be false.\\
\vspace{1mm}\\
\textbf{Example}\\
Suppose $P\to(Q\to R)$. Prove that $\neg R\to(P\to\neg Q)$.\\
\vspace{1mm}\\
\textit{Scratch work}\\
We start with the following situation:
\begin{center}
\begin{tabular}{c|c}
\textit{Givens}&\textit{Goal}\\
\hline
$P\to(Q\to R)$&$\neg R\to(P\to\neg Q)$\\
\end{tabular}
\end{center}
If either $P$ or $\neg(Q\to R)$ gets added to the givens list, then we should consider using modus ponens or modus tollens. For now we concentrate on the goal; being a conditional statement, 
we assume the antecedent and set the consequent as the new goal:
\begin{center}
\begin{tabular}{c|c}
\textit{Givens}&\textit{Goal}\\
\hline
$P\to(Q\to R)$&$P\to\neg Q$\\
$\neg R$&
\end{tabular}
\end{center}
Our proof currently looks like:\\
\vspace{1mm}\\
\indent Suppose $\neg R$.\\
\indent\indent [Proof of $P\to\neg Q$ goes here.]\\
\indent Therefore $\neg R\to(P\to\neg Q)$.\\
(next page)\newpage
\noindent\textbf{Example cont.}\\
We had
\begin{center}
\begin{tabular}{c|c}
\textit{Givens}&\textit{Goal}\\
\hline
$P\to(Q\to R)$&$P\to\neg Q$\\
$\neg R$&
\end{tabular}
\end{center}
The goal is still a conditional. We use the same strategy again:
\begin{center}
\begin{tabular}{c|c}
\textit{Givens}&\textit{Goal}\\
\hline
$P\to(Q\to R)$&$\neg Q$\\
$\neg R$&\\
$P$&
\end{tabular}
\end{center}
Now the proof looks like\\
\vspace{1mm}\\
\indent Suppose $\neg R$.\\
\indent\indent Suppose $P$.\\
\indent\indent\indent [Proof for $\neg Q$ goes here.]\\
\indent\indent Therefore $P\to\neg Q$.\\
\indent Therefore $\neg R\to(P\to\neg Q)$.\\
\vspace{1mm}\\
Since we know $P\to(Q\to R)$ and $P$, by modus ponens we can infer $Q\to R$:
\begin{center}
\begin{tabular}{c|c}
\textit{Givens}&\textit{Goal}\\
\hline
$P\to(Q\to R)$&$\neg Q$\\
$\neg R$&\\
$P$&\\
$Q\to R$&
\end{tabular}
\end{center}
We add one more line to our proof:\\
\vspace{1mm}\\
\indent Suppose $\neg R$.\\
\indent\indent Suppose $P$.\\
\indent\indent\indent Since $P$ and $P\to(Q\to R)$, it follows that $Q\to R$.\\
\indent\indent\indent [Proof of $\neg Q$ goes here.]\\
\indent\indent Therefore $P\to\neg Q$.\\
\indent Therefore $\neg R\to(P\to\neg Q)$.\\
\vspace{1mm}\\
Finally, our last step is to use modus tollens. We now know $Q\to R$ and $\neg R$, so by modus tollens we can conclude $\neg Q$.\\
(next page)\newpage
\noindent\textbf{Example cont.}\\
\textit{Solution}
\begin{theorem}
Suppose $P\to(Q\to R)$. Then $\neg R\to(P\to\neg Q)$.
\end{theorem}
\begin{proof}
Suppose $\neg R$. Suppose $P$. Since $P$ and $P\to(Q\to R)$, it follows that $Q\to R$. But then since $\neg R$, we can conclude $\neg Q$. Thus
$P\to\neg Q$. Therefore $\neg R\to(P\to\neg Q)$.
\end{proof}
\newpage

\section{Proofs involving quantifiers}
\subsection{Goals of form $\forall xP(x)$}
If you can give a proof of the goal $P(x)$ that would work for all $x$, then you can conclude that $\forall xP(x)$ must
be true. Thus it is important to start the proof with
no assumptions about $x$, that is, $x$ must be \textit{arbitrary}; one must not assume that $x$ is already equal
to any other object already under discussion in the proof.\\
\vspace{1mm}\\
If $x$ is already being used in the proof to stand for
some particular object, then one cannot use it to stand for an
arbitrary object, and must choose a different variable that is not already being used in the proof, say $y$, and replace 
$\forall xP(x)$ with the equivalent statement $\forall yP(y)$.\\
\vspace{1mm}\\
\indent To prove a goal of the form  $\forall P(x)$:\\
\indent Let $x$ stand for an arbitrary object and prove $P(x)$. The letter $x$ must be a new variable in the proof. 
If $x$ is already being used to stand for something, then choose an unused variable, say $y$, to stand for the arbitrary 
object, and prove $P(y)$.\\
\vspace{1mm}\\
\textit{Scratch work}\\
Before using strategy:
\begin{center}
\begin{tabular}{c|c}
\textit{Givens}&\textit{Goal}\\
\hline
---&$\forall xP(x)$\\
---&
\end{tabular}
\end{center}
After using strategy:
\begin{center}
\begin{tabular}{c|c}
\textit{Givens}&\textit{Goal}\\
\hline
---&$P(x)$\\
---&
\end{tabular}
\end{center}
\textit{Form of final proof:}\\
\indent Let $x$ be arbitrary.\\
\indent\indent [Proof of $P(x)$ goes here.]\\
\indent Since $x$ was arbitrary, we can conclude that $\forall xP(x)$.\\
(next page)\newpage
\noindent\textbf{Example}\\
Consider a proof covered earlier, but phrased slightly differently: Suppose $A,B$, and $C$ are sets, and $A\setminus B\subseteq C$. Prove that $A\setminus C\subseteq B$.\\
\vspace{1mm}\\
\textit{Scratch work}\\
Before using strategy:
\begin{center}
\begin{tabular}{c|c}
\textit{Givens}&\textit{Goal}\\
\hline
$A\setminus B\subseteq C$&$A\setminus C\subseteq B$
\end{tabular}
\end{center}
As usual we consider the logical form of the goal to plan our strategy. In this case we can write out the definition of $\subseteq$:
\begin{center}
\begin{tabular}{c|c}
\textit{Givens}&\textit{Goal}\\
\hline
$A\setminus B\subseteq C$&$\forall x(x\in A\setminus C\to x\in B)$
\end{tabular}
\end{center}
Because the goal has the form $\forall xP(x)$, where $P(x)$ is the statement $x\in A\setminus C\to x\in B$, we will introduce a new variable $x$ into the proof to stand for an arbitrary object
and then try to prove $x\in A\setminus C\to x\in B$.\\
\vspace{1mm}\\
Note that $x$ is a new variable in the proof; it appeared in the logical form of the goal as a bound variable, but, being a bound variable, didn't represent anything in particular. 
We also haven't used it as a free variable in any statement so it doesn't stand for any particular object. To make sure $x$ is arbitrary we must be careful not to add any assumptions
about $x$ to the givens column. Our goal becomes
\begin{center}
\begin{tabular}{c|c}
\textit{Givens}&\textit{Goal}\\
\hline
$A\setminus B\subseteq C$&$x\in A\setminus C\to x\in B$
\end{tabular}
\end{center}
Our proof looks like\\
\vspace{1mm}\\
\indent Let $x$ be arbitrary.\\
\indent\indent [Proof of $x\in A\setminus C\to x\in B$ goes here]\\
\indent Since $x$ was arbitrary, we can conclude that $\forall x(x\in A\setminus C\to x\in B)$,\\
\indent so $A\setminus C\subseteq B$.\\
\vspace{1mm}\\
The problem is now exactly the same as it was in the previous example.\\
\vspace{1mm}\\
\textit{Solution}
\begin{theorem}
Suppose $A,B$, and $C$ are sets, and $A\setminus B\subseteq C$. Then $A\setminus C\subseteq B$.
\end{theorem}
\begin{proof}
Let $x$ be arbitrary. Suppose $x\in A\setminus C$. This means that $x\in A$ and $x\notin C$. Suppose $x\notin B$. Then $x\in A\setminus B$, so since $A\setminus B\subseteq C$, $x\in C$. But
this contradicts the fact that $x\notin C$. Therefore $x\in B$. Thus, if $x\in A\setminus C$ then $x\in B$. Since $x$ was arbitrary we can conclude that $\forall x(x\in A\setminus C\to x\in B)$, so
$A\setminus C\subseteq B$.
\end{proof}
\noindent(next page)\newpage
\noindent\textbf{Another example}\\
Suppose $A$ and $B$ are sets. Prove that if $A\cap B=A$ then $A\subseteq B$.\\
\vspace{1mm}\\
\textit{Scratch work}\\
\vspace{1mm}\\
Our goal is $A\cap B=A\to A\subseteq B$. We add the antecedent to the givens list and make the consequent the goal. We write out the logical form of this new goal:
\begin{center}
\begin{tabular}{c|c}
\textit{Givens}&\textit{Goal}\\
\hline
$A\cap B=A$&$\forall x(x\in A\to x\in B)$
\end{tabular}
\end{center}
We let $x$ be arbitrary, assume $x\in A$, and probe $x\in B$:
\begin{center}
\begin{tabular}{c|c}
\textit{Givens}&\textit{Goal}\\
\hline
$A\cap B=A$&$x\in B$\\
$x\in A$&
\end{tabular}
\end{center}
Our final proof form looks like\\
\vspace{1mm}\\
\indent Suppose $A\cap B=A$.\\
\indent\indent Let $x$ be arbitrary.\\
\indent\indent\indent Suppose $x\in A$.\\
\indent\indent\indent\indent [Proof of $x\in B$ goes here.]\\
\indent\indent\indent Therefore $x\in A\to x\in B$.\\
\indent\indent Since $x$ was arbitrary, we can conclude that $\forall x(x\in A\to x\in B)$,\\
\indent\indent so $A\subseteq B$.\\
\indent Therefore, if $A\cap B=A$ then $A\subseteq B$.\\
\vspace{1mm}\\
Since $x\in A$ and $A\cap B=A$, it follows that $x\in A\cap B$, so $x\in B$. (recall $x\in A\cap B$ means $x\in A$ and $x\in B$)\\
\vspace{1mm}\\
Oftentimes the statement that $x$ is arbitrary is left out---when a new variable $x$ is introduced into a proof it is usually understood that it is arbitrary and no assumptions are being made 
about $x$. Many of the proofs written so far end with multiple concluding sentences; oftentimes these are condensed into a single sentence, or skipped entirely.\\
\vspace{1mm}\\
\textit{Solution}
\begin{theorem}
Suppose $A$ and $B$ are sets. If $A\cap B=A$ then $A\subseteq B$.
\end{theorem}
\begin{proof}
Suppose $A\cap B=A$, and suppose $x\in A$. Then since $A\cap B=A$, $x\in A\cap B$, so $x\in B$. Since $x$ was an arbitrary element of $A$, we can conclude that $A\subseteq B$.
\end{proof}
\newpage

\subsection{Goals of form $\exists xP(x)$}
Proving a goal of the form $\exists xP(x)$ also involves introducing a new variable $x$ into the proof and proving $P(x)$. But in this case $x$ will not be arbitrary; we only need to prove that 
$P(x)$ is true for \textit{at least one} $x$. It suffices to assign a particular value to $x$ and prove $P(x)$ for this one value of $x$.\\
\vspace{1mm}\\
\indent To prove a goal of the form $\exists xP(x)$:\\
\indent Try to find a value of $x$ for which $P(x)$ would be true. Start with `Let $x=\text{(decided value)}$' and proceed to prove $P(x)$ for this $x$. Once again if the variable name $x$ 
is already being used in the proof for some other purpose then one should choose an unused variable, say $y$ and rewrite the goal in the equivalent form $\exists yP(y)$ and proceed as before.\\
\vspace{1mm}\\
\textit{Scratch work}\\
Before using strategy:
\begin{center}
\begin{tabular}{c|c}
\textit{Givens}&\textit{Goal}\\
\hline
---&$\exists xP(x)$\\
---&
\end{tabular}
\end{center}
After using strategy:
\begin{center}
\begin{tabular}{c|c}
\textit{Givens}&\textit{Goal}\\
\hline
---&$P(x)$\\
---&\\
$x=\text{(the value you decided on)}$&
\end{tabular}
\end{center}
\textit{Form of final proof:}\\
\vspace{1mm}\\
\indent Let $x=\text{(the value you decided on)}$.\\
\indent\indent [Proof of $P(x)$ goes here.]\\
\indent Thus, $\exists xP(x)$.\\
\vspace{1mm}\\
Note that \textit{the reasoning used for finding a specific $x$ will not appear in the final proof}. To justify the conclusion that $\exists xP(x)$ it is only necessary to verify that $P(x)$ comes
out true when $x$ is assigned some particular value. How one thought of that value is not part of the justification of the conclusion (be it by trial and error etc.).\\
(next page)
\newpage
\noindent\textbf{Example}\\
Prove that for every real number $x$, if $x>0$ then there is a real number $y$ such that $y(y+1)=x$.\\
\vspace{1mm}\\
\textit{Scratch work}\\
Logically, our goal is $\forall x (x>0\to\exists y[y(y+1)=x])$, where the variables $x$ and $y$ in this statement are
understood to range over $\mathbb R$. We start by letting $x$ be an arbitrary real number, and then assume that $x>0$ and
try to prove that $\exists y[y(y+1)=x]$:
\begin{center}
\begin{tabular}{c|c}
\textit{Givens}&\textit{Goal}\\
\hline
$x>0$&$\exists y[y(y+1)=x]$
\end{tabular}
\end{center}
Because our goal has the form $\exists yP(y)$, where $P(y)$ is the statement $y(y+1)=x$, according to our strategy we should
try to find a value of $y$ for which $P(y)$ is true. In this case we do this by solving the equation $y(y+1)=x$ for $y$.
\begin{equation*}
y(y+1)=x\iff y^2+y-x=0\iff y=\frac{-1^2\pm\sqrt{1+4x}}{2}
\end{equation*}
Note that $\sqrt{1+4x}$ is defined since we have $x>0$ as a given. Also see that we have found two solutions for $y$, but to
prove that $\exists y[y(y+1)=x]$ we only need one valid $y$, so either of the the two solutions could be used. 
\begin{center}
\begin{tabular}{c|c}
\textit{Givens}&\textit{Goal}\\
\hline
$x>0$&$y(y+1)=x$\\
$y=(-1+\sqrt{1+4x})/2$&
\end{tabular}
\end{center}
Our final proof looks like\\
\vspace{1mm}\\
\indent Let $x$ be an arbitrary real number.\\
\indent\indent Suppose $x>0$.\\
\indent\indent\indent Let $y=(-1+\sqrt{1+4x})/2$.\\
\indent\indent\indent\indent [Proof of $y(y+1)=x$ goes here.]\\
\indent\indent\indent Thus, $\exists y[y(y+1)=x]$.\\
\indent\indent Therefore $x>0\to\exists y[y(y+1)=x]$.\\
\indent Since $x$ was arbitrary, we can conclude that $\forall x(x>0\to\exists y[y(y+1)=x])$.\\
\vspace{1mm}\\
(next page)\newpage
\noindent\textbf{Example cont.}\\
\textit{Solution}
\begin{theorem}
For every real number $x$, if $x>0$ then there is a real number $y$ such that $y(y+1)=x$.
\end{theorem}
\begin{proof}
Let $x$ be an arbitrary real number and suppose $x>0$. Let 
\begin{equation*}
y=\frac{-1+\sqrt{1+4x}}{2}
\end{equation*}
which is defined since $x>0$. Then
\begin{align*}
y(y+1)&=\left(\frac{-1+\sqrt{1+4x}}{2}\right)\cdot\left(\frac{-1+\sqrt{1+4x}}{2}+1\right)\\
&=\left(\frac{\sqrt{1+4x}-1}{2}\right)\cdot\left(\frac{\sqrt{1+4x}+1}{2}\right)\\
&=\frac{1+4x-1}{4}=\frac{4x}{4}=x\qedhere
\end{align*}
\end{proof}
\newpage

\subsection{Givens of the form $\exists xP(x)$ and $\forall xP(x)$}
\textbf{Givens of the form $\exists xP(x)$}\\
This given says that an object with a certain property exists. A good approach would be to introduce a new variable, say 
$x_0$, to stand for this object, and add $P(x_0)$ to the givens list.\\
\vspace{1mm}\\
\indent To use a given of the form $\exists xP(x)$:\\
\indent Introduce a new variable $x_0$ into the proof to stand for an object for which $P(x_0)$ is true. This means
that one can now assume that $P(x_0)$ is true.\\
\vspace{1mm}\\
Logicians call this \textit{existential instantiation}. Note that this is very different from our treatment of goals of the 
same form---we can assume that $x_0$ stands for some object for which $P(x_0)$ is true, but we can't assume anything else 
about $x_0$.\\
\vspace{1mm}\\
\textbf{Givens of the form $\forall xP(x)$}\\
This says that $P(x)$ would be true no matter what value is assigned ot $x$. One can therefore \textit{choose any value} to be
plugged in for $x$ and use this to conclude $P(x)$.\\
\vspace{1mm}\\
\indent To use a given of the form $\forall xP(x)$\\
\indent Plug in any value, say $a$, for $x$ and use this given to conclude that $P(a)$ is true.\\
\vspace{1mm}\\
This is called \textit{universal instantiation}. Usually, if we have a given of the form $\exists xP(x)$, we should apply
existential instantiation to it immediately. On the other hand,
we won't be able to apply universal instantiation to a given of the form $\forall xP(x)$ unless we had an idea for
a particular value $a$ to plug in for $x$; we would probably wait until a suitable object $a$ appears in the proof.\\
(next page)\newpage
\noindent\textbf{Example}\\
Suppose $\mathcal F$ and $\mathcal G$ are families of sets and $\mathcal F\cap\mathcal G\neq\emptyset$. Prove that 
$\bigcap\mathcal F\subseteq\bigcup\mathcal G$.\\
\vspace{1mm}\\
\textit{Scratch work}\\
First we analyse the logical form of the goal by writing out the meaning of the subset: 
$\forall x(x\in\bigcap\mathcal F\to x\in\bigcup\mathcal G)$. We could go further with the logical forms of the union and 
intersection, but the analysis done so far is sufficient for us to decide how to get started on the proof; their definitions
might be required later, but it is usually best to do only as much of the analysis as is needed to determine the next step
of the proof. Going further might introduce unnecessary complication.\\
\vspace{1mm}\\
We let $x$ be arbitrary, assume $x\in\bigcap\mathcal F$, and try to prove $x\in\bigcup\mathcal G$.
\begin{center}
\begin{tabular}{c|c}
\textit{Givens}&\textit{Goal}\\
\hline
$\mathcal F\cap\mathcal G\neq\emptyset$&$x\in\bigcup\mathcal G$\\
$x\in\bigcap\mathcal F$&
\end{tabular}
\end{center}
Now we further analyse the logical forms:
\begin{center}
\begin{tabular}{c|c}
\textit{Givens}&\textit{Goal}\\
\hline
$\exists A(A\in\mathcal F\cap\mathcal G)$&$\exists A\in\mathcal G(x\in A)$\\
$\forall A\in\mathcal F(x\in A)$&
\end{tabular}
\end{center}
To prove this new goal means we should try to find a value that would `work' for $A$. The second given starts with $\forall A$, so we may not be able to use it until a likely value for $A$ pops up
during the course of the proof. The first given, however, starts with $\exists A$, so we should use it immediately; by existential instantiation, we introduce a name, say $A_0$, for this object, 
and treat $A_0\in\mathcal F\cap\mathcal G$ as a given from now on. It would be redundant to continue to discuss the given statement $\exists A(A\in\mathcal F\cap\mathcal G)$ so we will drop it from 
our list of givens.
\begin{center}
\begin{tabular}{c|c}
\textit{Givens}&\textit{Goal}\\
\hline
$A_0\in\mathcal F$&$\exists A\in\mathcal G(x\in A)$\\
$A_0\in\mathcal G$&\\
$\forall A\in\mathcal F(x\in A)$&
\end{tabular}
\end{center}
An element to plug into the third given has now surfaced, plugging $A_0$ for $A$ we can conclude that $x\in A_0$; we can treat this as a given. See that the goal can now be easily reached.\\
\vspace{1mm}\\
Although we translated the statements $x\in\bigcap\mathcal F,x\in\bigcup\mathcal G$, and $\mathcal F\cap\mathcal G\neq\emptyset$ into logical symbols in order to figure out how to use
them in the proof; these translations are not usually written out it in the final proof---left to the reader to work out their logical forms in order to follow the reasoning.\\
(next page)\newpage
\noindent\textbf{Example cont.}\\
\textit{Solution}
\begin{theorem}
Suppose $\mathcal F$ and $\mathcal G $ are families of sets, and $\mathcal F\cap\mathcal G\neq\emptyset$. Then\\$\bigcap\mathcal F\subseteq\bigcup\mathcal G$.
\end{theorem}
\begin{proof}
Suppose $x\in\bigcap\mathcal F$. Since $\mathcal F\cap\mathcal G=\emptyset$, we can let $A_0$ be an element of $\mathcal F\cap\mathcal G$. Thus, $A_0\in\mathcal F$ and $A_0\in\mathcal G$. Since
$x\in\bigcap\mathcal F$ and $A_0\in\mathcal F$, it follows that $x\in A_0$. But we also know that $A_0\in\mathcal G$, so we can conclude that $x\in\bigcup\mathcal G$.
\end{proof}
\hspace{1mm}\\
\noindent\textbf{Another example}\\
Suppose $B$ is a set and $\mathcal F$ is a family of sets. Prove that if $\bigcup\mathcal F\subseteq B$ then $\mathcal F\subseteq\mathscr P(B)$\\
\vspace{1mm}\\
\textit{Scratch work}\\
We assume $\bigcup\mathcal F\subseteq B$ and try to prove $\mathcal F\subseteq\mathscr P(B)$, which means $\forall x(x\in\mathcal F\to x\in\mathscr P(B))$, we let $x$ be arbitrary, assume 
$x\in\mathcal F$, and set $x\in\mathscr P(B)$ as our goal. Note that $x$ in this case is a set:
\begin{center}
\begin{tabular}{c|c}
\textit{Givens}&\textit{Goal}\\
\hline
$\bigcup\mathcal F\subseteq B$&$x\in\mathscr P(B)$\\
$x\in\mathcal F$&\\
\end{tabular}
\end{center}
To further analyse this goal, we use the definition of the power statement $x\in\mathscr P(B)=x\subseteq B$; which further means $\forall y(y\in x\to y\in B)$. We must therefore introduce another
arbitrary object into the proof; we let $y$ be arbitrary, assume $y\in x$, and try to prove $y\in B$.
\begin{center}
\begin{tabular}{c|c}
\textit{Givens}&\textit{Goal}\\
\hline
$\bigcup\mathcal F\subseteq B$&$y\in B$\\
$x\in\mathcal F$&\\
$y\in x$&\\
\end{tabular}
\end{center}
The first given can be written as $\forall z(z\in\bigcup\mathcal F\to z\in B)$, which further means
$\forall z(\exists A\in\mathcal F(z\in A)\to z\in B)$
\begin{center}
\begin{tabular}{c|c}
\textit{Givens}&\textit{Goal}\\
\hline
$\forall z(\exists A\in\mathcal F(z\in A)\to z\in B)$&$y\in B$\\
$x\in\mathcal F$&\\
$y\in x$&\\
\end{tabular}
\end{center}
The conclusion is now easily reachable.\\
(next page)\newpage
\noindent\textbf{Example cont.}\\
\textit{Solution}
\begin{theorem}
Suppose $B$ is a set and $\mathcal F$ is a family of sets. If $\bigcup\mathcal F\subseteq B$ then $\mathcal F\subseteq\mathscr P(B)$.
\end{theorem}
\begin{proof}
Suppose $\bigcup\mathcal F\subseteq B$. Let $x$ be an arbitrary element of $\mathcal F$. Let $y$ be an arbitrary element of $x$. Since $y\in x$ and $x\in\mathcal F$, $y\in\bigcup\mathcal F$.
But since $\bigcup\mathcal F\subseteq B$, $y\in B$. Since $y$ was an arbitrary element of $x$, we can conclude that $x\subseteq B$, so $x\in\mathscr P(B)$. But $x$ was an arbitrary element of 
$\mathcal F$, so $\mathcal F\subseteq\mathscr P(B)$, as required.
\end{proof}

\subsection{Example: For all integers $a,b,c$, if $a\mid b$ and $b\mid c$ then $a\mid c$}
For this proof, we need the following definition:
\begin{definition}
For any integers $x$ and $y$, we'll say that \textit{$x$ divides $y$} (or \textit{$y$ is divisible by $x$}) if $\exists k\in\mathbb{Z}(kx=y)$. We use the notation $x\mid y$ to mean 
`$x$ divides $y$' and $x\nmid y$ to mean `$x$ does not divide $y$'.
\end{definition}
\begin{theorem}
For all integers $a,b$, and $c$, if $a\mid b$ and $b\mid c$ then $a\mid c$.
\end{theorem}
\begin{proof}
Let $a,b$, and $c$ be arbitrary integers and suppose $a\mid b$ and $b\mid c$. Since $a\mid b$, we can choose some integer $m$ such that $ma=b$. Similarly, since $b\mid c$, we can choose an integer
$n$ such that $nb=c$. Therefore $c=nb=nma$, so since $nm$ is an integer, $a\mid c$.
\end{proof}
\newpage

\section{Proofs involving conjunctions and biconditionals}






\appendix
\chapter{Exercises}
\section{Ch 3}
\subsection{$\exists x(P(x)\to Q(x))$ is equivalent to $\forall xP(x)\to\exists xQ(x)$}
Using logical equivalences:
\begin{align*}
\exists x(P(x)\to Q(x))&=\exists x[\neg(P(x)\land\neg Q(x))]&\text{(Definition)}\\
&=\neg\forall x[P(x)\land\neg Q(x)]&\text{(Quantifier negation)}\\
&=\neg[\forall xP(x)\land\forall x\neg Q(x)]&\text{(Distributivity over conjuction)}\\
&=\neg[\forall xP(x)\land\neg\exists xQ(x)]&\text{(Quantifier negation)}\\
&=\forall xP(x)\to\exists xQ(x)&\text{(Definition)}
\end{align*}

\subsection{If $\exists x(P(x)\to Q(x))$, then $\forall xP(x)\to\exists xQ(x)$}
\begin{proof}
Suppose $\exists x(P(x)\to Q(x))$. Then we can choose some $x_0$ such that $P(x_0)\to Q(x_0)$. Now suppose that $\forall xP(x)$.
Then in particular, $P(x_0)$, and since $P(x_0)\to Q(x_0)$, it follows that $Q(x_0)$. Since we have found a particular 
value of $x$ for which $Q(x)$ holds, we can conclude that $\exists xQ(x)$. Thus $\forall xP(x)\to\exists xQ(x)$.
\end{proof}
\newpage

\subsection{}









\end{document}
