\documentclass{report} 
\title{Velleman}
\date{Started 6th September 2025}
\author{Malcolm}
\usepackage{amsmath} %import math
\usepackage{mathtools} %more math
\usepackage{amssymb} %for QED symbol
\usepackage{amsthm} %
\usepackage{bm}%bold math
\usepackage{graphicx} %import imaging
\graphicspath{{./images/}} %set imaging path
\begin{document}
\maketitle

\tableofcontents

\newpage
\chapter{Logic}
\subsection{Logic Factsheet}
\textbf{De Morgan's laws}
\begin{align*}
&\neg\,(P\land Q)\text{ is equivalent to }\neg\,P\lor\neg\,Q\\
&\neg\,(P\lor Q)\text{ is equivalent to }\neg\,P\land\neg\,Q
\end{align*}
\textbf{Commutative laws}
\begin{align*}
P\land Q\text{ is equivalent to }Q\land P\\
P\lor Q\text{ is equivalent to }Q\lor P
\end{align*}
\textbf{Associative laws}
\begin{align*}
P\land(Q\land R)\text{ is equivalent to }(P\land Q)\land R\\
P\lor(Q\lor R)\text{ is equivalent to }(P\lor Q)\lor R
\end{align*}
\textbf{Indempotent laws}
\begin{align*}
P\land P\text{ is equivalent to }P\\
P\lor P\text{ is equivalent to }P
\end{align*}
\textbf{Distributive laws}
\begin{align*}
P\land(Q\lor R)\text{ is equivalent to }(P\land Q)\lor(P\land R)\\
P\lor(Q\land R)\text{ is equivalent to }(P\lor Q)\land(P\lor R)
\end{align*}
\textbf{Absorption laws}
\begin{align*}
P\lor(P\land Q)\text{ is equivalent to }P\\
P\land(P\lor Q)\text{ is equivalent to }P
\end{align*}
\textbf{Double Negation law}
\begin{align*}
\neg\neg\,P\text{ is equivalent to }P
\end{align*}
\newpage

\subsection{Set operation definitions}
The \textit{intersection} of two sets $A$ and $B$ is the set $A\cap B$ defined as follows:
\begin{align*}
A\cap B=\{x\,|\,x\in A\text{ and }x\in B\}
\end{align*}
The \textit{union} of $A$ and $B$ is the set $A\cup B$ defined as follows:
\begin{align*}
A\cup B=\{x\,|\,x\in A\text{ or }x\in B\}
\end{align*}
The \textit{difference} of $A$ and $B$ is the set $A\setminus B$ defined as follows:
\begin{align*}
A\setminus B=\{x\,|\,x\in A\text{ and }x\notin B\}
\end{align*}
See that 
\begin{equation*}
x\in A\cap B=x\in\{y\,|\,y\in A\text{ and }y\in B\}
\end{equation*}
where $y$ is a dummy variable. So we can also write that
\begin{equation*}
x\in A\cap B=x\in A\land x\in B
\end{equation*}
The same can be shown for the union and difference.

\subsection{Distributivity of set operations}
We show \begin{equation*} x\in A\cap(B\cup C)\text{ is equivalent to }
x\in(A\cap B)\cup(A\cap C)
\end{equation*}
By analysing their logical forms:
\begin{align*}
&x\in A\cap(B\cup C)\\
&=x\in A\land x\in(B\cup C)\\
&=x\in A\land(x\in B\lor x\in C)
\end{align*}
and
\begin{align*}
&x\in(A\cap B)\cup(A\cap C)\\
&=x\in(A\cap B)\lor x\in(A\cap C)\\
&=(x\in A\land x\in B)\lor(x\in A\land x\in C)\\
&=[(x\in A\land x\in B)\lor x\in A)]\land[(x\in A\land x\in B)\lor x\in C)]\\
&=x\in A\land[(x\in A\lor x\in C)\land(x\in B\lor x\in C)]\\
&=[x\in A\land(x\in A\lor x\in C)]\land(x\in B\lor x\in C)\\
&=x\in A\land(x\in B\lor x\in C)
\end{align*}
We can also show, in a similar manner, that
\begin{equation*}
x\in A\cup(B\cap C)\text{ is equivalent to }
x\in(A\cup B)\cap(A\cup C)
\end{equation*}
\newpage

\subsection{$x\in A\setminus(B\cap C)=x\in(A\setminus B)\cup(A\setminus C)$}
We can also show
\begin{equation*}
x\in A\setminus(B\cap C)=x\in(A\setminus B)\cup(A\setminus C)
\end{equation*}
See that
\begin{align*}
&x\in A\setminus(B\cap C)&\\
&=x\in A\land\neg\,(x\in B\cap C)&\text{(Definition of $\setminus$)}\\
&=x\in A\land\neg\,(x\in B\land x\in C)&\text{(Definition of $\cap$)}\\
&=x\in A\land(x\notin B\lor x\notin C)&\text{(De Morgan's)}\\
&=(x\in A\land x\notin B)\lor(x\in A\land x\notin C)&\text{(Distributivity)}\\
&=(x\in A\setminus B)\lor(x\in A\setminus C)&\text{(Definition of $\setminus$)}\\
&=x\in(A\setminus B)\cup(A\setminus C)&\text{(Definition of $\cup$)}
\end{align*}

\subsection{}







\end{document}
