\documentclass{report} 
\title{Velleman}
\date{Started 6th September 2025}
\author{Malcolm}
\usepackage{amsmath} %import math
\usepackage{mathtools} %more math
\usepackage{amssymb} %for QED symbol
\usepackage{amsthm} %
\usepackage{mathrsfs} %
\usepackage{bm}%bold math
\usepackage{graphicx} %import imaging

\graphicspath{{./images/}} %set imaging path
\newtheorem*{theorem}{Theorem}

\begin{document}
\maketitle

\tableofcontents

\newpage
\chapter{Logic}
\subsection{Logic Factsheet}
\textbf{De Morgan's laws}
\begin{align*}
&\neg\,(P\land Q)\text{ is equivalent to }\neg\,P\lor\neg\,Q\\
&\neg\,(P\lor Q)\text{ is equivalent to }\neg\,P\land\neg\,Q
\end{align*}
\textbf{Commutative laws}
\begin{align*}
P\land Q\text{ is equivalent to }Q\land P\\
P\lor Q\text{ is equivalent to }Q\lor P
\end{align*}
\textbf{Associative laws}
\begin{align*}
P\land(Q\land R)\text{ is equivalent to }(P\land Q)\land R\\
P\lor(Q\lor R)\text{ is equivalent to }(P\lor Q)\lor R
\end{align*}
\textbf{Indempotent laws}
\begin{align*}
P\land P\text{ is equivalent to }P\\
P\lor P\text{ is equivalent to }P
\end{align*}
\textbf{Distributive laws}
\begin{align*}
P\land(Q\lor R)\text{ is equivalent to }(P\land Q)\lor(P\land R)\\
P\lor(Q\land R)\text{ is equivalent to }(P\lor Q)\land(P\lor R)
\end{align*}
\textbf{Absorption laws}
\begin{align*}
P\lor(P\land Q)\text{ is equivalent to }P\\
P\land(P\lor Q)\text{ is equivalent to }P
\end{align*}
\textbf{Double Negation law}
\begin{align*}
\neg\neg\,P\text{ is equivalent to }P
\end{align*}
\newpage

\subsection{Set operation definitions}
The \textit{intersection} of two sets $A$ and $B$ is the set $A\cap B$ defined as follows:
\begin{align*}
A\cap B=\{x\,|\,x\in A\text{ and }x\in B\}
\end{align*}
The \textit{union} of $A$ and $B$ is the set $A\cup B$ defined as follows:
\begin{align*}
A\cup B=\{x\,|\,x\in A\text{ or }x\in B\}
\end{align*}
The \textit{difference} of $A$ and $B$ is the set $A\setminus B$ defined as follows:
\begin{align*}
A\setminus B=\{x\,|\,x\in A\text{ and }x\notin B\}
\end{align*}
See that 
\begin{equation*}
x\in A\cap B=x\in\{y\,|\,y\in A\text{ and }y\in B\}
\end{equation*}
where $y$ is a dummy variable. So we can also write that
\begin{equation*}
x\in A\cap B=x\in A\land x\in B
\end{equation*}
The same can be shown for the union and difference.

\subsection{Distributivity of set operations}
We show \begin{equation*} x\in A\cap(B\cup C)\text{ is equivalent to }
x\in(A\cap B)\cup(A\cap C)
\end{equation*}
By analysing their logical forms:
\begin{align*}
&x\in A\cap(B\cup C)\\
&=x\in A\land x\in(B\cup C)\\
&=x\in A\land(x\in B\lor x\in C)
\end{align*}
and
\begin{align*}
&x\in(A\cap B)\cup(A\cap C)\\
&=x\in(A\cap B)\lor x\in(A\cap C)\\
&=(x\in A\land x\in B)\lor(x\in A\land x\in C)\\
&=[(x\in A\land x\in B)\lor x\in A)]\land[(x\in A\land x\in B)\lor x\in C)]\\
&=x\in A\land[(x\in A\lor x\in C)\land(x\in B\lor x\in C)]\\
&=[x\in A\land(x\in A\lor x\in C)]\land(x\in B\lor x\in C)\\
&=x\in A\land(x\in B\lor x\in C)
\end{align*}
We can also show, in a similar manner, that
\begin{equation*}
x\in A\cup(B\cap C)\text{ is equivalent to }
x\in(A\cup B)\cap(A\cup C)
\end{equation*}
\newpage

\subsection{$x\in A\setminus(B\cap C)=x\in(A\setminus B)\cup(A\setminus C)$}
We can also show
\begin{equation*}
x\in A\setminus(B\cap C)=x\in(A\setminus B)\cup(A\setminus C)
\end{equation*}
See that
\begin{align*}
&x\in A\setminus(B\cap C)&\\
&=x\in A\land\neg\,(x\in B\cap C)&\text{(Definition of $\setminus$)}\\
&=x\in A\land\neg\,(x\in B\land x\in C)&\text{(Definition of $\cap$)}\\
&=x\in A\land(x\notin B\lor x\notin C)&\text{(De Morgan's)}\\
&=(x\in A\land x\notin B)\lor(x\in A\land x\notin C)&\text{(Distributivity)}\\
&=(x\in A\setminus B)\lor(x\in A\setminus C)&\text{(Definition of $\setminus$)}\\
&=x\in(A\setminus B)\cup(A\setminus C)&\text{(Definition of $\cup$)}
\end{align*}

\subsection{$x\in(A\cup B)\setminus(A\cap B)=x\in(A\setminus B)\cup(B\setminus A)$}
\begin{align*}
&x\in(A\cup B)\setminus(A\cap B)&\\
&=(x\in A\lor x\in B)\land\neg\,(x\in A\land x\in B)&(\text{By definition)}\\
&=(x\in A\lor x\in B)\land(x\notin A\lor x\notin B)&\text{(De Morgan's)}\\
&=[(x\in A\lor x\in B)\land(x\notin A)]&\\
&\qquad\qquad\qquad\lor[(x\in A\lor x\in B)\land(x\notin B)]&\text{(Distributivity)}\\
&=[(x\notin A\land x\in A)\lor(x\notin A\land x\in B)]&\\
&\qquad\qquad\qquad\lor[(x\notin B\land x\in A)\lor(x\notin B\land x\in B)]&\text{(Distributivity)}\\
&=(x\notin A\land x\in B)\lor(x\notin B\land x\in A)&\\
&=(x\in A\land x\notin B)\land(x\in B\land x\notin A)&\text{(Commutativity)}\\
&=x\in(A\setminus B)\cup(B\setminus A)&\text{(By definition)}
\end{align*}

\subsection{$(A\cap B)\cap(A\setminus B)=\emptyset$}
See that
\begin{align*}
&x\in(A\cap B)\cap(A\setminus B)&\\
&=(x\in A\land x\in B)\land(x\in A\land x\notin B)&\text{(Definition)}\\
&=x\in A\land\underbrace{(x\in B\land x\notin B)}_{\text{Contradiction}}&\text{(Associativity + Commutativity)}
\end{align*}
The last statement is a contradiction, so the statement 
$x\in(A\cap B)\cap(A\setminus B)$ will always be false, no matter what $x$ is. In other words, nothing can be an element of 
$(A\cap B)\cap(A\setminus B)$, so it must be the case that 
$(A\cap B)\cap(A\setminus B)=\emptyset$; $A\cap B$ and $A\setminus B$ are disjoint.
\newpage

\subsection{Conditional and Contrapositive laws}
\textbf{Conditional Law}
\begin{equation*}
P\to Q\text{ is equivalent to }\neg\,(P\land\neg\,Q)
\end{equation*}
by De Morgan's law we can also say that
\begin{equation*}
P\to Q\text{ is equivalent to }\neg\,P\lor Q
\end{equation*}
\textbf{Contrapositive law}
\begin{equation*}
P\to Q\text{ is equivalent to }\neg\,Q\to\neg\,P
\end{equation*}
This can be justified using
\begin{equation*}
P\to Q=\neg\,(P\land\neg\,Q)=\neg\,(\neg\,Q\land P)=\neg\,Q\to\neg\,P
\end{equation*}
\textbf{Intuition}\\
Intuitive ways to think of $P\to Q$ (and equivalently $\neg\,Q\to\neg\,P$) include:
\begin{itemize}
\item $P\text{ implies }Q$.
\item $Q\text{, if }P$.
\item $P\text{ only if }Q$.
\item $P\text{ is a sufficient condition for }Q$.
\item $Q\text{ is a necessary condition for }P$.
\end{itemize}
\newpage

\subsection{Biconditional statements}
We write
\begin{equation*}
P\leftrightarrow Q=(P\to Q)\land(Q\to P)
\end{equation*}
Note that by the contrapositive law, this is also equivalent to
\begin{equation*}
(P\to Q)\land(\neg\,P\to\neg\,Q)
\end{equation*}
\textbf{Intuition}\\
$Q\to P$ can be written as `$P$ if $Q$' and $P\to Q$ can be written as `$P$ only if $Q$' (since this means $\neg\,Q\to\neg\,P$ which is $P\to Q$).\\
\vspace{1mm}\\
Combining the two as $(P\to Q)\land(Q\to P)=P\leftrightarrow Q$ therefore corresponds to the statement 
`$P$ if and only if $Q$'.\\
\vspace{1mm}\\
$P\leftrightarrow Q$ means `$P$ iff $Q$', or `$P$ is a necessary and sufficient condition for $Q$'.
\newpage

\section{Quantificational logic}
\subsection{Quantifier negation laws}
We have
\begin{align*}
&\neg\exists xP(x)\text{ is equivalent to }\forall x\neg P(x)\\
&\neg\forall xP(x)\text{ is equivalent to }\exists x\neg P(x)
\end{align*}
\textbf{Intuition}\\
No matter what $P(x)$ stands for, the formula $\neg\exists xP(x)$ means that there is no value of $x$ for which $P(x)$ is true; this is ths same as saying that for every value of $x$ in 
the universe of discourse, $P(x)$ is false---meaning $\forall x\neg P(x)$.\\
\vspace{1mm}\\
Similarly, to say that $\neg\forall xP(x)$ means that it is not the case that for all values of $x$, $P(x)$ is true. This is equivalent to saying that that there is at least one
value of $x$ for which $P(x)$ is false---so $\exists x\neg P(x)$.

\subsection{Notation}
\textbf{`Exactly one' notation}\\
We write 
\begin{equation*}
\exists!xP(x)=\exists x(P(x)\land\neg\exists y(P(y)\land y\neq x))
\end{equation*}
As a shorthand way to write `there is exactly one value of $x$ such that $P(x)$ is true', or `there is a unique $x$ such that $P(x)$'.\\
\vspace{1mm}\\
\textbf{Specifying quantifiers}\\
We write
\begin{equation*}
\forall x\in A\, P(x)
\end{equation*}
to mean that \textit{for every value of $x$ in the set $A$, $P(x)$ is true}. Similarly,
\begin{equation*}
\exists x\in A\, P(x)
\end{equation*}
means \textit{there is at least one value of $x $ in the set $A$ such that $P(x)$ is true}.\\
\vspace{1mm}\\
Formulas containing bounded quantifiers can also be thought of as abbreviations for more complicated formulas containing only normal, unbounded quantifiers. See that 
\begin{equation*}
\forall x\in A\, P(x)=\forall x(x\in A\to P(x))
\end{equation*}
and
\begin{equation*}
\exists x\in A\,P(x)=\exists x(x\in A\land P(x))
\end{equation*}
\newpage

\subsection{Negation law for bounded quantifiers}
We can show
\begin{equation*}
\neg\forall x\in A\,P(x)=\exists x\in A\,\neg P(x)
\end{equation*}
See that
\begin{align*}
&\neg\forall x\in A\,P(x)&\\
&=\neg\forall x(x\in A\to P(x))&\text{(as defined)}\\
&=\exists x\neg(x\in A\to P(x))&\text{(negation law)}\\
&=\exists x\neg\neg(x\in A\land\neg P(x))&\text{(conditional law)}\\
&=\exists x(x\in A\land\neg P(x))&\\
&=\exists x\in A\,\neg P(x)&\text{(as defined)}
\end{align*}
Similarly we can show
\begin{equation*}
\neg\exists x\in A\,P(x)=\forall x\in A\,\neg P(x)
\end{equation*}
See that
\begin{align*}
&\neg\exists x\in A\,P(x)&\\
&=\neg\exists x(x\in A\land P(x))&\text{(as defined)}\\
&=\forall x\neg(x\in A\land P(x))&\text{(negation law)}\\
&=\forall x(x\in A\to\neg P(x))&\text{(conditional law)}\\
&=\forall x\in A\,\neg P(x)&\text{(as defined)}
\end{align*}
\newpage

\subsection{Vacuously true}
It is clear that if $A=\emptyset$ then $\exists x\in A\,P(x)$ will be false regardless of $P(x)$, since there is nothing in $A$ that makes $P(x)$ come true 
(since there is nothing in $A$ to being with).\\
\vspace{1mm}\\
Now consider $\forall x\in A\,P(x)$. We can reason that
\begin{equation*}
\forall x\in A\,P(x)=\neg\exists x\in A\,\neg P(x)\quad\text{(quantifier negation)}
\end{equation*}
See that if $A=\emptyset$ then this formula will be true, no matter what $P(x)$ is. In this case we say that the statement is \textit{vacuously true}.\\
\vspace{1mm}\\
Another way to see this is to rewrite
\begin{equation*}
\forall x\in A\,P(x)=\forall x(x\in A\to P(x))
\end{equation*}
The only way this can be false is if there is some value of $x$ such that $x\in A$ is true but $P(x)$ false; but there is no such value of $x$.
Intuitively, because the condition cannot be met, it is impossible to provide a counterexample to prove something wrong.\\
\vspace{1mm}\\
An analogy would be me claiming `i've never lost a race to Usain Bolt'. This is true, but vacuously so.
\newpage

\subsection{Alternate definition for indexed families}
Say we are looking for the set $\{p_1,p_2,\ldots,p_{100}\}$; another way of describing this set would be to say that it consists of all numbers $p_i$, for $i$ an element of the set
$I=\{1,2,3,\ldots,100\}=\{i\in\mathbb{N}|1\leq i\leq 100\}$. We can write
\begin{equation*}
P=\{p_i|i\in I\}
\end{equation*}
Each element $p_i$ in this set is identified by $i\in I$, called the \textit{index} of each element. A set defined this way is called an \textit{indexed family}, and $I$ the \textit{index set}.
Although the indices for an indexed family are often numbers, they need not be.\\
\vspace{1mm}\\
In general, see that any indexed family
\begin{equation*}
A=\{x_i|i\in I\}
\end{equation*}
Can also be defined as
\begin{equation*}
A=\{x|\exists i\in I\,(x=x_i)\}
\end{equation*}
It follows that the statement
\begin{equation*}
x\in\{x_i|i\in I\}
\end{equation*}
means the same thing as
\begin{equation*}
\exists i\in I\,(x=x_i)
\end{equation*}
\newpage

\subsection{Power set}
Suppose $A$ is as set. The \textit{power set} of $A$, denoted $\mathscr{P}(A)$, is the set whose elements are all subsets of $A$. In other words,
\begin{equation*}
\mathscr{P}(A)=\{x|x\subseteq A\}
\end{equation*}
For instance, the set $A=\{7,12\}$ has four subsets $\emptyset,\{7\},\{12\}$, and $\{7,12\}$;
thus, $\mathscr{P}(A)=\{\emptyset,\{7\},\{12\},\{7,12\}\}$. 

\subsection{Intersection and union of a family of sets}
Suppose $\mathcal{F}$ is a family of sets. The \textit{intersection} and \textit{union} of $\mathcal{F}$ are the sets $\bigcap\mathcal{F}$ and $\bigcup\mathcal{F}$ are defined as follows:
\begin{align*}
&\bigcap\mathcal F=\{x|\forall A\in\mathcal F(x\in A)\}=\{x|\forall A(A\in\mathcal F\to x\in A)\}\\
&\bigcup\mathcal F=\{x|\exists A\in\mathcal F(x\in A)\}=\{x|\exists A(A\in\mathcal F\land x\in A)\}
\end{align*}
Notice that if $A$ and $B$ are any two sets and $\mathcal F=\{A,B\}$, then $\bigcap\mathcal F=A\cap B$ and $\bigcup\mathcal F=A\cup B$; the definitions of intersection and union of a family of sets
are generalisations of our old definitions of the intersection and union of two sets.\\
\vspace{1mm}\\
\textbf{Alternative notation}\\
An alternative notation is sometimes used for the union or intersection of an indexed family of sets. Suppose $\mathcal F=\{A_i|i\in I\}$, where each $A_i$ is a set,
then $\bigcap\mathcal F$ and $\bigcup\mathcal F$ could also be written as 
$\bigcap_{i\in I}A_i$ and $\bigcup_{i\in I}A_i$; as such
\begin{align*}
&\bigcap\mathcal F=\bigcap_{i\in I}A_i=\{x|\forall i\in I(x\in A_i)\}\\
&\bigcup\mathcal F=\bigcup_{i\in I}A_i=\{x|\exists i\in I(x\in A_i)\}
\end{align*}
\newpage

\subsection{More on set notation}
One generally defines a set using the elementhood test notation
\begin{equation*}
\{x|P(x)\}
\end{equation*}
Where the set consists of all $x$ that satisfy the specified condition $P(x)$. Sometimes this notation can be modified to allow the $x$ before the vertical line to be replaced
with a more complex expression. For example, suppose we wanted to define $S$ to be the set of all
perfect squres, we could write
\begin{equation*}
S=\{n^2|n\in\mathbb{N}\}
\end{equation*}
This is the same as
\begin{equation*}
S=\{x|\exists n\in\mathbb N(x=n^2)\}
\end{equation*}
See therefore that 
\begin{equation*}
x\in\{n^2|n\in\mathbb{N}\}=\exists n\in\mathbb N(x=n^2)
\end{equation*}

\chapter{Proof Strategies}
\subsection{Terminology}
We want to state the answer to a mathematical question in the form of a \textit{theorem} that says that if certain assumptions called the \textit{hypotheses} of the theorem are true, then
some conclusion must also be true.\\
\vspace{1mm}\\
An assignment of particular values to these variables is called an \textit{instance} of the theorem, and in order for the theorem to be correct it must be the case
that for every instance of the theorem that makes the hypotheses come out true, the conclusion is also true.\\
\vspace{1mm}\\
If there is even one instance
in which the hypotheses are true but the conclusion is false, then the theorem is incorrect; such an instance is called a \textit{counterexample} to the theorem.\\
\vspace{1mm}\\
As in the next section, we will refer to statements that are known or assumed to be true at some point in the course of figuring out the proof as \textit{givens}, and the statements that remains
to be proven at that point as the \textit{goal}.
\newpage

\subsection{To prove a conclusion of the form $P\to Q$}
To prove a conclusion of the form $P\to Q$, we can \textit{assume $P$ is true and then prove $Q$}.\\
\vspace{1mm}\\
Assuming that $P$ is true amounts to adding $P$ to the lists of hypotheses. If the conclusion of the theorem we are trying to prove has the form $P\to Q$, then we can \textit{transform the
problem} by adding $P$ to the list of hypotheses and changing the conclusion form $P\to Q$ to $Q$.\\
\vspace{1mm}\\
How we solve this new problem will now then be guided by the logical form of the new conclusion $Q$, and perhaps also that of the new hypothesis $P$.\\
\vspace{1mm}\\
This strategy is one that proves the \textit{goal} of $P\to Q$. Even if the conclusion of a theorem is not a conditional statement, if we transform the problem in such a way that the conditional
statement becomes the goal, then we can apply this strategy as the next step in figuring out the proof.\\
\vspace{1mm}\\
\textbf{Example:}
\begin{theorem}
Suppose $a$ and $b$ are real numbers. If $0<a<b$ then $a^2<b^2$.
\end{theorem}
\begin{proof}
Suppose $0<a<b$. Multiplying the inequality $a<b$ by the positive number $a$ we can conclude that $a^2<ab$; similarly multiplying by $b$ we get $ab<b^2$. Therefore $a^2<ab<b^2$, so $a^2<b^2$.
\end{proof}
\noindent We were given as a hypothesis that $a$ and $b$ are real numbers with a conclusion of the form $P\to Q$, where $P$ is the statement $0<a<b$ and $Q$ the statement $a^2<b^2$. Thus
we start with these statements as given and goal:
\begin{center}
\begin{tabular}{c|c}
\textit{Givens}&\textit{Goal}\\
$a$ and $b$ are real numbers&$(0<a<b)\to(a^2<b^2)$
\end{tabular}
\end{center}
As per our strategy, we assume that $0<a<b$ and try to use this assumption to prove $a^2<b^2$. In other words we add $0<a<b$ to the list of givenas and make $a^2<b^2$ our goal:
\begin{center}
\begin{tabular}{c|c}
\textit{Givens}&\textit{Goal}\\
$a$ and $b$ are real numbers&$a^2<b^2$\\
$0<a<b$&
\end{tabular}
\end{center}





\end{document}
